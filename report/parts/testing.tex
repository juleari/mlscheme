\section{Тестирование}
    \subsection{Удобство использования языка}
    В таблице~\ref{tabular:code_scheme} приведены реализации одних и тех же функций на Языке и на Scheme.
    Как видно из таблицы, код на Языке легче читается из-за отстутствия большого количества скобок, привычной инфиксной нотации и сопоставления с образцом.
    Язык позволяет достичь более краткой записи при заметно меньшем числе повторяющихся элементов синтаксиса.
    Использование повторных переменных в примере №2 позволяет существенно улучшить восприятие кода.
    \begin{table}[ht!]
        \caption{Соответствие конструкций Языка конструкциям языка Scheme}
        \centering
        \label{tabular:code_scheme}
        \begin{tabular}{|l|l|}
            \hline
            \bf{Входной язык}            & \bf{Scheme} \\ \hline
            \verb,n! 0 <- 1,             & \verb,(define (n! n), \\
            \verb,n! n <- n * n! (n - 1),& \verb,  (if (zero? n),\\
                                         & \verb,      1, \\
                                         & \verb,      (* n (n! (- n 1))))), \\ \hline

            \verb,count x []         <- 0,     & \verb,(define (count x xs), \\
            \verb,count x [ x : xs ] <-,       & \verb,  (if (null? xs), \\
            \verb,    1 + count x xs,          & \verb,      0, \\
            \verb,count x [ y : xs ] <-,       & \verb,      (if (equal? x (car xs)), \\
            \verb,    count x xs,              & \verb,          (+ 1 (count x (cdr xs))), \\
                                               & \verb,          (count x (cdr xs))))), \\ \hline
            
            \verb,cycle xs 0 <- [],            & \verb,(define (cycle xs n), \\
            \verb,cycle xs n <- ,              & \verb,  (if (zero? n),\\
            \verb,    xs ++ cycle xs ( n - 1 ),& \verb,      '(), \\
                                               & \verb,      (append xs, \\
                                               & \verb,              (cycle xs, \\
                                               & \verb,                     (- n 1))))), \\ \hline
            % \verb,nil? 0  <- #t,               & \verb,(define (nil? x), \\
            % \verb,nil? [] <- #t,               & \verb,  (or (zero? x),\\
            % \verb,nil? x  <- #f,               & \verb,      (null? x))),\\ \hline
            
            \verb,day-of-week day month year <-,& \verb,(define (day-of-week day month year), \\
            \verb,    a <- (14 - month) // 12,  & \verb,  (let* ((a (quotient (- 14 month) 12)),\\
            \verb,    y <- year - a,            & \verb,         (y (- year a)), \\
            \verb,    m <- month + (a * 12) - 2,& \verb,         (m (- (+ month (* a 12)) 2))), \\
                                                & \\
            \verb,    (7000 + day + y,          & \verb,    (remainder (- (+ 7000 day y, \\
            \verb,          + (y // 4),         & \verb,                     (quotient y 4), \\
            \verb,          + (y // 400),       & \verb,                     (quotient y 400),\\
            \verb,          + (31 * m // 12),   & \verb,                     (quotient (* 31 m) 12)),\\
            \verb,          - (y // 100)) % 7,  & \verb,                     (quotient y 100)), \\
                                                & \verb,                     7))), \\ \hline
        \end{tabular}
    \end{table}
    \clearpage
    \subsection{Корректность работы программ}

    Тестирование корректности работы программы представлено в таблице~\ref{tabular:corr}.

    \LTcapwidth=\textwidth
    \begin{longtable}[ht!]{|l|l|l|c|}
        %\LTcapwidth{5in}
        \caption[plates vibr-ortotrop-edge]{Тестирование корректности работы программы}\label{tabular:corr}\\
        %\multicolumn{2}{c}{Ключевые слова и зарезервированные символы} \\
        %\begin{tabular}
            \hline
            \bf{Код программы,}             & \bf{Вывод}         & \bf{Результат}  & \bf{Статус} \\
            \bf{на исходном языке}          & \bf{компилятора}   & \bf{выполнения} & \\ \hline\endhead

            \verb,day-of-week day month year <-,& \verb,LEXER OK!,   & 2 & ok \\
            \verb,    a <- (14 - month) // 12,  &                    & 0 & \\
            \verb,    y <- year - a,            & \verb,SYNTAX OK!,  & 2 & \\
            \verb,    m <- month + (a * 12) - 2,&                    & 3 & \\
                                                & \verb,SEMANTIC OK!,& & \\
            \verb,    (7000 + day + y ,         &                    & & \\
            \verb,          + (y // 4),         &                    & & \\
            \verb,          + (y // 400),       &                    & & \\
            \verb,          + (31 * m // 12) ,  &                    & & \\
            \verb,          - (y // 100)) % 7,  &                    & & \\
                                                &                    & & \\
            \verb,day-of-week 17 5 2016,        &                    & & \\
            \verb,day-of-week 10 4 2016,        &                    & & \\
            \verb,day-of-week 29 3 2016,        &                    & & \\
            \verb,day-of-week 20 4 2016,        &                    & & \\ \hline

            \verb,n! 0 <- 1,                & \verb,LEXER OK!,   &                 & ok \\
            \verb,n! n <- n * n! (n - 1),   &                    &                 & \\
                                            & \verb,SYNTAX OK!,  &                 & \\
            \verb,n! 5,                     &                    & \verb,120,      & \\
            \verb,n! 10,                    & \verb,SEMANTIC OK!,& \verb,3628800,  & \\ \hline

            \pagebreak
        
            \verb,replace pred? proc [] <- [],& \verb,LEXER OK!,   & & ok \\
            \verb,replace pred?,              &                    & & \\
            \verb,        proc ,              & \verb,SYNTAX OK!,  & & \\
            \verb,        [ x : xs ] <-,      &                    & & \\
            \verb,  if | pred? x ->,          & \verb,SEMANTIC OK!,& & \\
            \verb,       [ proc.[x] :,        &                    & & \\
            \verb,         replace pred? proc xs ], &                    & & \\
            \verb,     |         -> ,               &                    & & \\
            \verb,       [ x :,                     &                    & & \\
            \verb,       replace pred? proc xs ],   &                    & & \\  
                                                &                    & & \\
            \verb,replace zero?,                &                    & [1 1 2 3 0] & \\
            \verb,        \ x -> x + 1,         &                    & & \\
            \verb,        [ 0 1 2 3 0 ],        &                    & & \\
                                                &                    & & \\
            \verb,replace odd?,                 &                    & [2 2 6 4 10 6] & \\
            \verb,        \ x -> x * 2,         &                    & & \\
            \verb,        [ 1 2 3 4 5 6 ],      &                    & & \\
                                                &                    & & \\
            \verb,replace \ x -> 0 > x,         &                    & [0 1 & \\
            \verb,        exp,                  &                    &  0.3678794 2& \\
            \verb,        [ 0 1 -1 2 -2 3 -3],  &                    &  0.1353352 3& \\
                                                &                    &  0.0497870]& \\ \hline

            \pagebreak

            \verb,sum <- 0,                  & \verb,LEXER OK!,   &                 & ok \\
            \verb,sum x : xs <- x + sum . xs,&                    &                 & \\
                                             & \verb,SYNTAX OK!,  &                 & \\
            \verb,sum 1 2 3 4,               &                    & \verb,10,       & \\
            \verb,sum.[5 6 7 8 9 10],        & \verb,SEMANTIC OK!,& \verb,45,       & \\ \hline

            \verb,count x []         <- 0,  & \verb,LEXER OK!,   & & ok \\
            \verb,count x [ x : xs ] <-,    &                    & & \\
            \verb,    1 + count x xs,       & \verb,SYNTAX OK!,  & & \\
            \verb,count x [ y : xs ] <-,    &                    & & \\
            \verb,    count x xs,           & \verb,SEMANTIC OK!,& & \\
                                            &                    & & \\
            \verb,count 1 [1 2 3 1],        &                    & 2 & \\
            \verb,count 0 [1 2 3 4],        &                    & 0 & \\
            \verb,count 5 [1 2 3 4 5 5 5],  &                    & 3 & \\ \hline
        %\end{tabular}
    \end{longtable}

    \subsection{Производительность}
        \subsubsection{Производительность работы компилятора}
        \subsubsection{Производительность работы программы}
        \subsubsection{Производительность при мемоизации}

        Тестирование производительности при мемоизации производилось для функции вычисления \verb,n,~-~ного числа фибоначчи.

        Сравнение результатов на рисунке~\ref{pic:memo}.

        \begin{figure}[ht!]
            \centering
            \begin{tikzpicture}[scale=1]
                \begin{axis}[ylabel=время выполнения (c.), xlabel=число n,
                ] %\tiny
                    \addplot coordinates {
                        (1, 0.000000)
                        (5, 0.000000)
                        (10, 0.000000)
                        (15, 0.000000)
                        (20, 0.010000)
                        (25, 0.010000)
                        (30, 0.130000)
                        (35, 1.410000)
                        (40, 15.440000)
                        (45, 176.380000)
                        (50, 1851.720000)
                    };
                    \addplot coordinates {
                        (1, 0.010000)
                        (5, 0.000000)
                        (10, 0.000000)
                        (15, 0.000000)
                        (20, 0.010000)
                        (25, 0.000000)
                        (30, 0.000000)
                        (35, 0.000000)
                        (40, 0.000000)
                        (45, 0.000000)
                        (50, 0.000000)
                    };
                \end{axis}
            \end{tikzpicture}
            \caption{График тестирования производительности при мемоизации}
            \label{pic:memo}
        \end{figure}

    На графике синим цветом обозначена зависимость времени выполнения функции вычисления \verb,n,~-~ного числа ряда Фибоначчи от числа \verb,n,, без мемоизации.
    Красным цветом --- c мемоизацией.

    Как видно из графика мемоизация существенно увеличивает скорость вычислений для числа \verb,n, большего 40.

    Сравнение проведено с помощью \verb;,time; для версии компилятора \verb,guile (GNU Guile) 2.0.11,
