% Введение, актуальность -- 1
\section*{Введение}
    Ранее, язык программирования был воплощением какой-либо парадигмы, концепции.
    Сейчас --- при разработке языка стремятся добиться удобства.
    Особое внимание уделяется краткости, удобству написания и чтения кода.

    К настоящему времени разработано несколько тысяч языков программирования, из которых около двадцати являются широко распространёнными\cite{TIOBE}.
    У всех есть свои сильные стороны и под конкретные задачи выбирается конкретный язык программирования, наиболее подходящий для этих целей.
    Так, например, для написания кросс-платформенных приложений используются Java или C++\cite{p_c_lisp}.
    Для браузерных расширений и сайтов --- JavaScript.

    Однако, для большинства языков, выигрыш от их выбора не так очевиден.
    Пользователи языка Python считают код на своём языке простым для понимания, потому что чтение программы на Python напоминает чтение текста на английском языке.
    Это позволяет сосредоточится на решении задачи, а не на самом языке.
    Lisp является программируемым языком программирования, что позволяет изменять и дополнять его под конкретные задачи.
    Язык OCaml благодаря системе вывода типов позволяет писать высокоэффективные и безопасные приложения.

    Анализ текущего состояния в разработке языков программирования показал, что существует необходимость в языке программирования, ориентированном на быстрой разработке сценариев (скриптов), первоначальном обучении программированию и исследовательском программировании (когда изначально неизвестно как программа должна работать).

    Для этих целей, на наш взгляд, должен существовать функциональный язык с <<дружелюбным>> синтаксисом, который подразумевает инфиксную нотацию в записи арифметических выражений.
