\section*{\hfillПРИЛОЖЕНИЕ\hfill}
\addcontentsline{toc}{section}{Приложение}
    \subsection*{ПРИЛОЖЕНИЕ А. Функция вычисления дня недели}
    \addcontentsline{toc}{subsection}{Приложение А. Функция вычисления дня недели}
    \lstinputlisting{../examples/1.sm}
    В этом программе определена функция вычисления дня недели по дате.

    В этом примере используются определения внутри функции.
    Так, три переменные \verb,a,, \verb,y, и \verb,m, определены и доступны только внутри функции \verb,day-of-week,.

    В процессе выполнения такой программы будут возвращены четыре значения: 2 (вторник), 0 (воскресенье), 2 (вторник), 3 (среда).

    \subsection*{ПРИЛОЖЕНИЕ Б. Функция замены всех нулей на единицы}
    \addcontentsline{toc}{subsection}{Приложение Б. Функция замены всех нулей на единицы}
    \lstinputlisting{../examples/2.sm}

    В этом примере используется сопоставление с образцом.
    Если аргумент функции --- пустой список, то она возвращает пустой список.
    Если первым элементом списка является число 0, то возвращается список, первым элементом которого является 1, а остаток этого списка вычисляется рекурсивно.
    Если шаблоны, описанные в первых двух строках не подошли, будет использован шаблон в третьей строке.
    Он более универсален.
    Результатом вычисления в этом случае будет список, состоящий из значения аргумента \verb,x, и списка, вычисленного рекурсивно.

    Вычисленные в ходе выполнения программы списки выглядят следующим образом:
    \\\verb,[1 2 7 1 5],
    \\\verb,[1 1 1 1 1],

    \subsection*{ПРИЛОЖЕНИЕ В. Функция подсчёта количества вхождений}
    \addcontentsline{toc}{subsection}{Приложение В. Функция подсчёта количества вхождений}
    \lstinputlisting{../examples/3.sm}

    Функция \verb,count, возвращает количество вхождений первого аргумента, в список, являющийся вторым аргументом.
    
    В первой строке мы видим определение функции \verb,count, от \verb,x, и пустого списка.
    В этом случае количество вхождений \verb,x, в список равно нулю.
    Во второй строке мы видим определение функции \verb,count, от \verb,x, и списка, первым элементом которого является \verb,x,.
    В данном случае нам нужно прибавить единицу к количеству вхождений \verb,x, в <<хвост>> списка \verb,xs,.
    И, наконец, в третьей строке определена функция \verb,count, от \verb,x, и списка, первым элементом которого является \verb,y,.
    Так как в строке выше мы предусмотрели равенство \verb,x, и первого элемента списка, то в этой строке определения мы можем быть уверены, что \verb,x, не равен \verb,y,.

    Результат выполнения программы:
    \\\verb,2,
    \\\verb,0,
    \\\verb,3,
    \subsection*{ПРИЛОЖЕНИЕ Г. Функция вычисления факториала числа}
    \addcontentsline{toc}{subsection}{Приложение Г. Функция вычисления факториала числа}
    \lstinputlisting{../examples/4.sm}

    \subsection*{ПРИЛОЖЕНИЕ Д. Функция вычисления суммы произвольного количества аргументов}
    \addcontentsline{toc}{subsection}{Приложение Д. Функция вычисления суммы произвольного количества аргументов}
    \lstinputlisting{../examples/5.sm}

    В данном примере определена функция вычисления факториала числа.
    Запись этой функции похожа на математическую запись этой функции.
    В случае равенства первого аргумента нулю --- будет возвращена единица,
    иначе будет посчитано произведение числа \verb,n, на факториал от числа \verb,n - 1,.

    В четвёртой строке программы указано, что эту функцию необходимо мемоизировать.

    Значение, вычисленное для числа 5 --- 120.
    Значение, вычисленное для числа 10 --- 3628800.

    Стоит отметить, что при вычислении факториала 10, высчитываются только значения этой функции для чисел 9, 8, 7, 6.
    Значение для числа 5 уже вычисленно при первом вызове функции.

    \subsection*{ПРИЛОЖЕНИЕ Е. Функция замены элементов списка}
    \addcontentsline{toc}{subsection}{Приложение Е. Функция замены элементов списка}
    \lstinputlisting{../examples/6.sm}

    В этой программе описана функция \verb,replace,, принимающая три аргумента:
    \begin{itemize}
        \item предикат (\verb,pred?,) --- функцию, проверяющую какое-либо условие;
        \item функцию одного аргумента (\verb,proc,), с помощью которой будет осуществлено преобразование;
        \item список.
    \end{itemize}

    В случае, если список пуст, возвращается пустой список.

    Для списка, не являющегося пустым будет осуществлена проверка первого элемента с помощью предиката.
    В случае, если условие, описанное в предикате выполнено, будет возвращён список, первым элементом которого будет являться изменённый функцией \verb,proc, первый элемент списка (\verb,x,).
    <<Хвост>> итогового списка вычисляется рекурсивно для списка \verb,xs,.

    С шестой по восьмую строках программы записан вызов функции \verb,replace, от стандартного предиката \verb,zero?, (проверяющего, является ли входной аргумент нулём), $\lambda$-функции, увеличивающей число на единицу и списка из 5 элементов.
    Результатом этого вызова будет список \verb,[1 1 2 3 0],.

    С десятой до двенадцатую строки программы описывается вызов функции \verb,replace, от стандартной функции \verb,odd?, (осущестляющей проверку на нечётность), $\lambda$-функции, удваивающей свой аргумент и списка из шести чисел.
    Результатом этого вызова будет список \verb,[1 1 2 3 0],.

    С четырнадцатой по шестнадцатую строку программы дано описание вызова функции от $\lambda$-функции, проверяющей, является ли число большим нуля, функции вычисления экспоненты числа и списка из семи аргументов.
    Результат этого вызова: \verb,[0 1 0.3678794 2 0.1353352 3 0.0497870],.

    \subsection*{ПРИЛОЖЕНИЕ Ё. Функция повтора элемента}
    \addcontentsline{toc}{subsection}{Приложение Ё. Функция повтора элемента}
    \lstinputlisting{../examples/7.sm}

    Результат выполнения программы:
    \\ \verb,["a" "a" "a" "a" "a"],
    \\ \verb,[["a" "b"] ["a" "b"] ["a" "b"]],
    \\ \verb,[],

    \subsection*{ПРИЛОЖЕНИЕ Ж. Функция повторения списка}
    \addcontentsline{toc}{subsection}{Приложение Ж. Функция повторения списка}
    \lstinputlisting{../examples/8.sm}

    Результат выполнения программы:
    \\ \verb,[0 1 0 1 0 1],
    \\ \verb,["a" "b" "c" "a" "b" "c" "a" "b" "c" "a" "b" "c" "a" "b" "c"],
    \\ \verb,[],

    \subsection*{ПРИЛОЖЕНИЕ З. Функция AND для произвольного количества аргументов}
    \addcontentsline{toc}{subsection}{Приложение З. Функция AND для произвольного количества аргументов}
    \lstinputlisting{../examples/9.sm}

    Результат выполнения программы:
    \\ \verb,#f,
    \\ \verb,#f,
    \\ \verb,#f,
    \\ \verb,#t,
    \\ \verb,#t,

    \subsection*{ПРИЛОЖЕНИЕ И. Вставка кода на языке Scheme}
    \addcontentsline{toc}{subsection}{Приложение И. Вставка кода на языке Scheme}
    Пример вставки кода на языке Scheme.
    Вставляемые функции --- функции сортировки выбором и вставками.
    \lstinputlisting{../examples/10.sm}

    Результат выполнения программы:
    \\ \verb,[0 1 2 3 4 5 6 7 8 9],
    \\ \verb,[0 1 2 3 4 5 6 7 8 9],

    \subsection*{ПРИЛОЖЕНИЕ К. Функция проверки числа на простоту}
    \addcontentsline{toc}{subsection}{Приложение К. Функция проверки числа на простоту}
    \lstinputlisting{../examples/11.sm}

    Результат выполнения программы:
    \\ \verb,#t,
    \\ \verb,#f,
    \\ \verb,#t,

    \subsection*{ПРИЛОЖЕНИЕ Л. Функции НОД и НОК}
    \addcontentsline{toc}{subsection}{Приложение Л. Функции НОД и НОК}
    \lstinputlisting{../examples/12.sm}

    Результат выполнения программы:
    \\ \verb,154,
    \\ \verb,12,

    \subsection*{ПРИЛОЖЕНИЕ М. Метод половинного сечения}
    \addcontentsline{toc}{subsection}{Приложение М. Метод половинного сечения}
    \lstinputlisting{../examples/13.sm}

    Результат выполнения программы:
    \\ \verb,-1.5703125,

    \subsection*{ПРИЛОЖЕНИЕ Н. Функция вычисляющая диапазон значений}
    \addcontentsline{toc}{subsection}{Приложение Н. Функция вычисляющая диапазон значений}
    \lstinputlisting{../examples/14a.sm}

    Результат выполнения программы:
    \\ \verb,[0 3 6 9],

    \subsection*{ПРИЛОЖЕНИЕ О. Функция разворачивания списков}
    \addcontentsline{toc}{subsection}{Приложение О. Функция разворачивания списков}
    \lstinputlisting{../examples/15.sm}

    Результат выполнения программы:
    \\ \verb,[1 2 3 4 5 6 7 8 9],

    \subsection*{ПРИЛОЖЕНИЕ П. Функция проверки вхождения в список}
    \addcontentsline{toc}{subsection}{Приложение П. Функция проверки вхождения в список}
    \lstinputlisting{../examples/16a.sm}

    Результат выполнения программы:
    \\ \verb,#t,
    \\ \verb,#f,

    