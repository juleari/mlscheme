\section*{\hfillПРИЛОЖЕНИЕ\hfill}
\addcontentsline{toc}{section}{Приложение}
    \subsection*{ПРИЛОЖЕНИЕ А. Функция вычисления дня недели}
    \addcontentsline{toc}{subsection}{Приложение А. Функция вычисления дня недели}
    \lstinputlisting{../examples/1.sm}
    В этом программе определена функция вычисления дня недели по дате.

    В этом примере используются определения внутри функции.
    Так, три переменные \verb,a,, \verb,y, и \verb,m, определены и доступны только внутри функции \verb,day-of-week,.

    \subsection*{ПРИЛОЖЕНИЕ Б. Функция замены всех нулей на единицы}
    \addcontentsline{toc}{subsection}{Приложение Б. Функция замены всех нулей на единицы}
    \lstinputlisting{../examples/2.sm}

    В этом примере используется сопоставление с образцом.
    Если аргумент функции --- пустой список, то она возвращает пустой список.
    Если первым элементом списка является число 0, то возвращается список, первым элементом которого является 1, а остаток этого списка вычисляется рекурсивно.
    Если шаблоны, описанные в первых двух строках не подошли, будет использован шаблон в третьей строке.
    Он более универсален.
    Результатом вычисления в этом случае будет список, состоящий из значения аргумента \verb,x, и списка, вычисленного рекурсивно.

    \subsection*{ПРИЛОЖЕНИЕ В. Функция подсчёта количества вхождений}
    \addcontentsline{toc}{subsection}{Приложение В. Функция подсчёта количества вхождений}
    \lstinputlisting{../examples/3.sm}

    \subsection*{ПРИЛОЖЕНИЕ Г. Функция вычисления факториала числа}
    \addcontentsline{toc}{subsection}{Приложение Г. Функция вычисления факториала числа}
    \lstinputlisting{../examples/4.sm}

    \subsection*{ПРИЛОЖЕНИЕ Д. Функция вычисления суммы произвольного количества аргументов}
    \addcontentsline{toc}{subsection}{Приложение Д. Функция вычисления суммы произвольного количества аргументов}
    \lstinputlisting{../examples/5.sm}

    \subsection*{ПРИЛОЖЕНИЕ Е. Функция замены элементов списка}
    \addcontentsline{toc}{subsection}{Приложение Е. Функция замены элементов списка}
    \lstinputlisting{../examples/6.sm}

    \subsection*{ПРИЛОЖЕНИЕ Ё. Функция повтора элемента}
    \addcontentsline{toc}{subsection}{Приложение Ё. Функция повтора элемента}
    \lstinputlisting{../examples/7.sm}

    \subsection*{ПРИЛОЖЕНИЕ Ж. Функция повторения списка}
    \addcontentsline{toc}{subsection}{Приложение Ж. Функция повторения списка}
    \lstinputlisting{../examples/8.sm}

    \subsection*{ПРИЛОЖЕНИЕ З. Функция AND для произвольного количества аргументов}
    \addcontentsline{toc}{subsection}{Приложение З. Функция AND для произвольного количества аргументов}
    \lstinputlisting{../examples/9.sm}

    \subsection*{ПРИЛОЖЕНИЕ И. Вставка кода на языке Scheme}
    \addcontentsline{toc}{subsection}{Приложение И. Вставка кода на языке Scheme}
    \lstinputlisting{../examples/10.sm}

    \subsection*{ПРИЛОЖЕНИЕ К. Функция проверки числа на простоту}
    \addcontentsline{toc}{subsection}{Приложение К. Функция проверки числа на простоту}
    \lstinputlisting{../examples/11.sm}

    \subsection*{ПРИЛОЖЕНИЕ Л. Функции НОД и НОК}
    \addcontentsline{toc}{subsection}{Приложение Л. Функции НОД и НОК}
    \lstinputlisting{../examples/12.sm}

    \subsection*{ПРИЛОЖЕНИЕ М. Метод половинного сечения}
    \addcontentsline{toc}{subsection}{Приложение М. Метод половинного сечения}
    \lstinputlisting{../examples/13.sm}

    \subsection*{ПРИЛОЖЕНИЕ Н. Функция вычисляющая диапазон значений}
    \addcontentsline{toc}{subsection}{Приложение Н. Функция вычисляющая диапазон значений}
    \lstinputlisting{../examples/14а.sm}

    \subsection*{ПРИЛОЖЕНИЕ О. Функция разворачивания списков}
    \addcontentsline{toc}{subsection}{Приложение О. Функция разворачивания списков}
    \lstinputlisting{../examples/15.sm}

    \subsection*{ПРИЛОЖЕНИЕ П. Функция проверки вхождения в список}
    \addcontentsline{toc}{subsection}{Приложение П. Функция проверки вхождения в список}
    \lstinputlisting{../examples/16a.sm}

    