\begin{thebibliography}{0}
\addcontentsline{toc}{section}{Список литературы}
    \bibitem{p_c_lisp}Peter Seibel. Practical Common Lisp. Publication Date: April 6, 2005
    \bibitem{scheme_doc}http://www.schemers.org/Documents/Standards/R5RS/HTML/
    \bibitem{scheme_pl}R. Kent Dybvig. The Scheme Programming Language, Fourth Edition. The MIT Press. 2009
    \bibitem{prolog}Information technology — Programming languages — Prolog. 1995
    \bibitem{python}A Byte of Python, http://python.swaroopch.com/
    \bibitem{ocaml}Yaron Minsky, Anil Madhavapeddy, Jason Hickey. Real world OCaml
    \bibitem{wiki_prolog}{title={Пролог},url={https://ru.wikipedia.org/wiki/Пролог_(язык_программирования)},type={online}}
    
    \bibitem{skor} Скоробогатов С.Ю. Лекции по курсу 'Компиляторы', 2014.
    \bibitem{economic_sajin} Арсеньев В.В., Сажин Ю.Б. Методические указания к выполнению организационно-экономической части дипломных проектов по созданию программной продукции. М.: изд. МГТУ им. Баумана, 1994. 52 с. 2.
    \bibitem{economic_smirnov} Под ред. Смирнова С.В. Организационно-экономическая часть дипломных проектов исследовательского профиля. М.: изд. МГТУ им. Баумана, 1995. 100 с.
\end{thebibliography}
