\begin{thebibliography}{0}
    \bibitem{TIOBE}\verb$http://www.tiobe.com/tiobe_index$
    \bibitem{p_c_lisp}Peter Seibel. Practical Common Lisp. Publication Date: April 6, 2005
    \bibitem{scheme_doc}\verb$http://www.schemers.org/Documents/Standards/R5RS/HTML/$
    \bibitem{scheme_pl}R. Kent Dybvig. The Scheme Programming Language, Fourth Edition. The MIT Press. 2009
    \bibitem{prolog}Information technology — Programming languages — Prolog. 1995
    \bibitem{unify}http://www.nsl.com/misc/papers/martelli-montanari.pdf
    \bibitem{prolog_bratko}Братко, Иван. Алгоритмы искусственного интеллекта на языке PROLOG, 3-е издание. Пер. с англ. --- М. : Издательский дом <<Вильяме>>, 2004.
    \bibitem{python}A Byte of Python, http://python.swaroopch.com/
    \bibitem{ocaml}Yaron Minsky, Anil Madhavapeddy, Jason Hickey. Real world OCaml. 2013
    \bibitem{haskell}Miran Lipovača. LEARN YOU A HASKELL FOR GREAT GOOD!. 2011 

    \bibitem{evm} Орлов С.А., Цилькер Б.Я. Организация ЭВМ и систем: Учебник для вузов. 2-е изд. --- СПб.: Питер, 2011
    \bibitem{langs} Сергей Александрович Орлов. Теория и практика языков программирования: Учебник для вузов. Стандарт 3-го поколения. --- СПб.: Питер, 2014
    \bibitem{skor} Скоробогатов С.Ю. Лекции по курсу <<Компиляторы>>, 2014.
    \bibitem{economic_sajin} Арсеньев В.В., Сажин Ю.Б. Методические указания к выполнению организационно-экономической части дипломных проектов по созданию программной продукции. М.: изд. МГТУ им. Баумана, 1994. 52 с. 2.
    %\bibitem{economic_smirnov} Под ред. Смирнова С.В. Организационно-экономическая часть дипломных проектов исследовательского профиля. М.: изд. МГТУ им. Баумана, 1995. 100 с.
\end{thebibliography}
