% Синтаксис ФЯП -- 3
\section{Синтаксис функционального языка программирования}
%\addcontentsline{toc}{section}{Синтаксис функционального языка программирования}
    Для описания синтаксиса языка используются расширенная форма Бэкуса-Наура (РБНФ). 
    Альтернатива обозначается символом '|'. 
    '*' после выражения означает, что оно может быть включено $0$ и более раз, '+' - $1$ и более, '?' - $0$ или $1$ раз.\cite{skor}
    Нетерминальные символы начинаются с заглавной буквы. 
    Терминальные либо начинаются малой буквой, либо состоят целиком из заглавных букв.
    Правила записываются в виде \verb!Идентификатор ::= выражение!,
    где идентификатор -- нетерминал, а выражение -- соответствующая правилам РБНФ комбинация терминальных и нетерминальных символов и специальных знаков.
    Точка в конце — специальный символ, указывающий на завершение правила.

    Язык является регистрозависимым.

    \subsection{Словарь и представление}
        Существуют следующие виды токенов: идентификатор, число, строка, операторы и ключевые слова.
        Пробельные символы игнорируются, если они не существенны для разделения двух последовательных токенов.

        \begin{itemize}
            \item Идентификаторы -- последовательности символов, исключающие пробельные символы, точки, запятые, скобки, кавычки, вертикальную черту.
            В отличие от большинства языков идентификаторы могут начинаться с цифр.
            В этом случае в состав этого идентификатора должен входить хотя бы один символ, не являющийся точкой или буквой \verb!e!.

            Примеры идентификаторов: \verb#day-of-week#, \verb#0->1#, \verb#nil?#, \verb#%2!0?#.
            \item Число -- целочисленная, вещественная константа, дробь или комплексное число.
            Число (в том числе дробь и вещественное число) может быть записано в шестнадцатеричной системе счисления.
            
            Примеры чисел: \verb!#xA9.F!, \verb!4/7!, \verb#2+7i#, \verb!#x7/ad!.
            \lstinputlisting{syntax/number}
            \item Строки -- последовательности символов, заключённые в двойные(") кавычки.
            
            Примеры строк: \verb#"valid string"#, \verb#"it's a beautiful day"#.
            \item Операторы и ключевые слова -- специальные символы, пары символов и слова, зарезервированные системой.

            Список таких символов: 
            
            \verb,( ) [ ] { } . : \ < <= > >= ! != = ^,

            \verb,- + ++ / // % * ** || && <- -> #t #f,

            Список таких слов: \verb!scheme!, \verb!mod!, \verb!if!, \verb!zero?!, \verb!eval!, \verb!abs!, \verb!odd?!, \verb!even?!, \verb!div!, \verb!round!, \verb!reverse!,
             \verb!null?!, \verb!not!, \verb!sin!, \verb!cos!, \verb!tg!, \verb!ctg!, \verb!eq?!, \verb!eqv?!, \verb!equal?!, \verb!gcd!, \verb!lcm!, \verb!expt!, \verb!sqrt!.
        \end{itemize}

    \subsection{Описание функций}
        