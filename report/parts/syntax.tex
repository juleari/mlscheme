% Синтаксис ФЯП -- 3
\section{Синтаксис функционального языка программирования}
%\addcontentsline{toc}{section}{Синтаксис функционального языка программирования}
    Для описания синтаксиса языка используются расширенная форма Бэкуса-Наура (РБНФ). 
    Альтернатива обозначается символом '|'. 
    '*' после выражения означает, что оно может быть включено $0$ и более раз, '+' - $1$ и более, '?' - $0$ или $1$ раз%\cite{skor}.
    Нетерминальные символы начинаются с заглавной буквы. 
    Терминальные либо начинаются малой буквой, либо состоят целиком из заглавных букв.

    Язык является регистрозависимым.

    \subsection{Словарь и представление}
        Существуют следующие виды токенов: идентификатор, число, символ, строка, операторы и ключевые слова.
        Пробельные символы игнорируются, если они не существенны для разделения двух последовательных токенов.

        \begin{itemize}
            \item Идентификаторы -- последовательности символов, исключающие пробельные символы, точки, запятые, скобки, кавычки, вертикальную черту.
            Второй вариант записи идентификаторов -- внутри двух вертикальных черт -- подразумевает возможность использовать любые символы.

            \lstinputlisting{../syntax/ident}
            \item Число -- целочисленная или вещественная константа. Типом числа считается минимальный тип, содержащий значение этого числа. 
            Если константа начинается с $'0x'$, то число рассматривается, как записанное в шестнадцатеричной системе счисления. 
            Иначе - в десятичной.
            
            \lstinputlisting{../syntax/number}
            \item Строки -- последовательности символов, заключённые в одинарные (') или двойные(") кавычки.
                
            %\lstinputlisting{lst_str}
            \item Операторы и ключевые слова -- специальные символы, пары символов и слова, зарезервированные системой.

            Список таких символов: $ + - < > * / = ~ , " ' . : | [ ] ( ) { } \ -- -> <-$ if scheme map filter reduce eval
        \end{itemize}