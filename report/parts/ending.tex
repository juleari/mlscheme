\conclusiontitle
    В ходе данной работы был предложен функциональный язык
    программирования с динамической типизацией и ML-подобным синтаксисом
    и разработан компилятор этого языка.

    В качестве языков-прототипов выступили функциональные языки Scheme, Haskell, объектно-функциональный язык OCaml и язык логического программирования Prolog. 
    В результате анализа синтаксиса, семантики и основных конструкций и возможностей этих языков был выработан оригинальный язык с синтаксисом, близким к языку Haskell и с семантикой, наиболее близкой к языку Scheme. 
    Программы на этом языке не перегружены элементами синтаксиса и ключевыми словами, выражения записываются в нотации, близкой к принятой в математике. 
    На наш взгляд, это способствует удобству использования разработанного языка для быстрого написания коротких программ (скриптов, сценариев), а также для обучения программированию.

    В язык введено сопоставление с образцом, которое расширено использованием повторных переменных.
    Применение таких переменных в образцах является новым для функциональных языков программирования.
    В ходе выполнения работы на примере было показано, что эта возможность органично дополняет функциональный язык и позволяет писать короткие и выразительные определения функций.
    При реализации генератора кода было обнаружено, что эта особенность не приводит к значительным накладным расходам при выполнении программ на целевом языке.
    Таким образом, эта конструкция может быть рекомендована к использованию при разработке новых и дополнения существующих функциональных языков программирования.

    Указание выполнить мемоизацию результатов вычисления функции с помощью единственного ключевого слова, введенного в язык, также благоприятно влияет на краткость и выразительность кода.

    Разработанный компилятор может быть использован на различных платформах совместно с различными реализациями языка Scheme, поддерживающих спецификацию R5RS. 

    Имеется возможность сборки компилятора в виде исполнимого файла, хотя это требует дополнения кода интерфейсом командной стоки, который специфичен для различных реализаций языка Scheme.

    Исходя из изложенного, цель работы была достигнута.

    В ходе работы было показано, что использование одного и того же языка одновременно в качестве основного прототипа, целевого языка трансляции и основного языка разработки компилятора может быть удачным решением с точки зрения экономии времени при разработке первой версии компилятора нового языка.
    Также следует отметить, что несмотря на возраст, язык Scheme по-прежнему является гибким, современным и удобным инструментом для исследовательского программирования и быстрого прототипирования приложений.
       