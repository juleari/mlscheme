\section{Технико-экономическое обоснование}
    Разработка программного обеспечения~---~достаточно трудоемкий и длительный процесс, требующий выполнения большого числа разнообразных операций.
    Организация и планирование процесса разработки программного продукта или программного комплекса при традиционном методе планирования предусматривает выполнение следующих работ:
    \begin{itemize}
        \item формирование состава выполняемых работ и группировка их по стадиям разработки;
        \item расчет трудоемкости выполнения работ;
        \item установление профессионального состава и расчет количества исполнителей;
        \item определение продолжительности выполнения отдельных этапов разработки;
        \item построение календарного графика выполнения разработки;
        \item контроль выполнения календарного графика.
    \end{itemize}

    Трудоемкость разработки программной продукции зависит от ряда факторов, основными из которых являются следующие: степень новизны разрабатываемого программного комплекса, сложность алгоритма его функционирования, объем используемой информации, вид ее представления и способ обработки, а также уровень используемого алгоритмического языка программирования.
    Чем выше уровень языка, тем трудоемкость меньше.

    По степени новизны разрабатываемый проект относится к \textit{группе новизны В} – разработка программной продукции, имеющей аналоги.

    По степени сложности алгоритма функционирования проект относится к \textit{3 группе сложности} - программная продукция, реализующая алгоритмы стандартных методов решения задач.

    По виду представления исходной информации и способа ее контроля программный продукт относится к \textit{группе 12} - исходная информация представлена в форме документов, имеющих различный формат и структуру и \textit{группе 22} - требуется печать документов одинаковой формы и содержания, вывод массивов данных на машинные носители.



    \subsection{Трудоемкость разработки программной продукции}
    \label{subsec:trud}
        Трудоемкость разработки программной продукции~($\tau_{PP}$) может быть определена как сумма величин трудоемкости выполнения отдельных стадий разработки программного продукта из выражения:
        $$\tau_{PP} = \tau_{TZ} + \tau_{EP} + \tau_{TP} + \tau_{RP} + \tau_{V},$$
        где $\tau_{TZ}$~---~трудоемкость разработки технического задания на создание программного продукта;
        $\tau_{EP}$~---~трудоемкость разработки эскизного проекта программного продукта;
        $\tau_{TP}$~---~трудоемкость разработки технического проекта программного продукта;
        $\tau_{RP}$~---~трудоемкость разработки рабочего проекта программного продукта;
        $\tau_{V}$~---~трудоемкость внедрения разработанного программного продукта.

        \subsubsection{Трудоемкость разработки технического задания}
            Расчёт трудоёмкости разработки технического задания~($\tau_{TZ}$)~[чел.-дни] производится по формуле:
            $$\tau_{TZ} = T^Z_{RZ} + T^Z_{RP},$$
            где $T^Z_{RZ}$~---~затраты времени разработчика постановки задачи на разработку ТЗ,~[чел.-дни];
            $T^Z_{RP}$~---~затраты времени разработчика программного обеспечения на разработку ТЗ,~[чел.-дни].
            Их значения рассчитываются по формулам:
            $$T^Z_{RZ} = t_Z * K^Z_{RZ},$$
            $$T^Z_{RP} = t_Z * K^Z_{RP},$$
            где $t_Z$~--~норма времени на разработку ТЗ на программный продукт~(зависит от функционального назначения и степени новизны разрабатываемого программного продукта),~[чел.-дни].
            В нашем случае по таблице получаем значение~(группа новизны – В, функциональное назначение – технико-экономическое):
            $$t_Z = 37.$$
            $K^Z_{RZ}$~---~коэффициент, учитывающий удельный вес трудоемкости работ, выполняемых разработчиком постановки задачи на стадии ТЗ.
            В нашем случае~(совместная разработка с разработчиком ПО):
            $$K^Z_{RZ} = 0.65.$$
            $K^Z_{RP}$~---~коэффициент, учитывающий удельный вес трудоемкости работ, выполняемых разработчиком программного обеспечения на стадии ТЗ.
            В нашем случае~(совместная разработка с разработчиком постановки задач):
            $$K^Z_{RP} = 0.35.$$
            Тогда:
            $$\tau_{TZ} = 37 *~(0.35 + 0.65) = 37.$$

        \subsubsection{Трудоемкость разработки эскизного проекта}
            Расчёт трудоёмкости разработки эскизного проекта~($\tau_{EP}$)~[чел.-дни] производится по формуле:
            $$\tau_{EP} = T^E_{RZ} + T^E_{RP},$$
            где $T^E_{RZ}$~---~затраты времени разработчика постановки задачи на разработку эскизного проекта~(ЭП),~[чел.-дни];
            $T^E_{RP}$~---~затраты времени разработчика программного обеспечения на разработку ЭП,~[чел.-дни].
            Их значения рассчитываются по формулам:
            $$T^E_{RZ} = t_E * K^E_{RZ},$$
            $$T^E_{RP} = t_E * K^E_{RP},$$
            где $t_E$~--~норма времени на разработку ЭП на программный продукт~(зависит от функционального назначения и степени новизны разрабатываемого программного продукта),~[чел.-дни].
            В нашем случае по таблице получаем значение~(группа новизны – В, функциональное назначение – технико-экономическое):
            $$t_E = 77.$$
            $K^E_{RZ}$~---~коэффициент, учитывающий удельный вес трудоемкости работ, выполняемых разработчиком постановки задачи на стадии ЭП.
            В нашем случае~(совместная разработка с разработчиком ПО):
            $$K^E_{RZ} = 0.7.$$
            $K^E_{RP}$~---~коэффициент, учитывающий удельный вес трудоемкости работ, выполняемых разработчиком программного обеспечения на стадии ТЗ.
            В нашем случае~(совместная разработка с разработчиком постановки задач):
            $$K^E_{RP} = 0.3.$$
            Тогда:
            $$\tau_{EP} = 77 *~(0.3 + 0.7) = 77.$$

        \subsubsection{Трудоемкость разработки технического проекта}
            Трудоёмкость разработки технического проекта~($\tau_{TP}$)~[чел.-дни] зависит от функционального назначения программного продукта, количества разновидностей форм входной и выходной информации и определяется по формуле:
            $$\tau_{TP} = (t^T_{RZ} + t^T_{RP})*K_V*K_R,$$
            где $t^T_{RZ}$~---~норма времени, затрачиваемого на разработку технического проекта~(ТП) разработчиком постановки задач,~[чел.-дни];
            $t^T_{RP}$~---~норма времени, затрачиваемого на разработку ТП разработчиком ПО,~[чел.-дни].
            По таблице принимаем~(функциональное назначение~---~технико-экономическое планирование,
            количество разновидностей форм входной информации~---~1~(файл с текстом программы на исходном языке),
            количество разновидностей форм выходной информации~---~1~(файл с текстом программы на языке Scheme)):
            $$t^T_{RZ} = 30,$$
            $$t^T_{RP} = 8.$$
            $K_R$~---~коэффициент учета режима обработки информации. По таблице принимаем~(группа новизны~---~В, режим обработки информации~---~реальный масштаб времени):
            $$K_R = 1.26.$$
            $K_V$~---~коэффициент учета вида используемой информации, определяется по формуле:
            $$K_V = \dfrac {K_P*n_P + K_{NS}*n_{NS} + K_B*n_B} {n_P + n_{NS} + n_B },$$
            где $K_P$~---~коэффициент учета вида используемой информации для переменной информации;
            $K_{NS}$~---~коэффициент учета вида используемой информации для нормативно-справочной информации;
            $K_B$~---~коэффициент учета вида используемой информации для баз данных;
            $n_P$~---~количество наборов данных переменной информации;
            $n_{NS}$~---~количество наборов данных нормативно-справочной информации;
            $n_B$~---~количество баз данных.
            Коэффициенты находим по таблице~(группа новизны - В):
            $$K_P=1.00,$$
            $$K_{NS}=0.72,$$
            $$K_B=2.08.$$
            Количество наборов данных, используемых в рамках задачи:
            $$n_P=10,$$
            $$n_{NS}=0,$$
            $$n_B=0.$$
            Находим значение $K_V$:
            $$K_V = \dfrac{1.00*10+0.72*0+2.08*1}{10+0+1}=1.098.$$
            Тогда:
            $$\tau_{TP} = (30+8)*1.098*1.26 = 53.$$

        \subsubsection{Трудоемкость разработки рабочего проекта}
            Трудоёмкость разработки рабочего проекта~($\tau_{RP}$)~[чел.-дни] зависит от функционального назначения программного продукта, количества разновидностей форм входной и выходной информации, сложности алгоритма функционирования, сложности контроля информации, степени использования готовых программных модулей, уровня алгоритмического языка программирования и определяется по формуле:
            $$\tau_{RP} = (t^R_{RZ} + t^R_{RP})*K_K*K_R*K_Y*K_Z*K_{IA},$$
            где $t^R_{RZ}$~---~норма времени, затраченного на разработку рабочего проекта на алгоритмическом языке высокого уровня разработчиком постановки задач,~[чел.-дни].
            $t^R_{RP}$~---~норма времени, затраченного на разработку рабочего проекта на алгоритмическом языке высокого уровня разработчиком ПО,~[чел.-дни].
            По таблице принимаем~(функциональное назначение~---~технико-экономическое планирование,
            количество разновидностей форм входной информации~---~1~(файл с текстом программы на исходном языке),
            количество разновидностей форм выходной информации~---~1~(файл с текстом программы на языке Scheme)):
            $$t^R_{RZ} = 8,$$
            $$t^R_{RP} = 51.$$
            $K_K$~---~коэффициент учета сложности контроля информации.
            По таблице принимаем~(степень сложности контроля входной информации~---~12, степень сложности контроля выходной информации~---~22):
            $$K_K = 1.00.$$
            $K_R$~---~коэффициент учета режима обработки информации.
            По таблице принимаем~(группа новизны~---~В, режим обработки информации~---~реальный масштаб времени):
            $$K_R = 1.26.$$
            $K_Y$~---~коэффициент учета уровня используемого алгоритмического языка программирования. По таблице принимаем значение~(интерпретаторы, языковые описатели):
            $$K_Y = 0.8.$$
            $K_Z$~---~коэффициент учета степени использования готовых программных модулей. По таблице принимаем~(использование готовых программных модулей составляет менее 25%%):
            $$K_Z = 0.8.$$
            $K_{IA}$~---~коэффициент учета вида используемой информации и сложности алгоритма программного продукта, его значение определяется по формуле:
            $$K_IA = \dfrac {K'_P*n_P + K'_{NS}*n_{NS} + K'_B*n_B} {n_P + n_{NS} + n_B },$$
            где $K'_P$~---~коэффициент учета сложности алгоритма ПП и вида используемой информации для переменной информации;
            $K'_{NS}$~---~коэффициент учета сложности алгоритма ПП и вида используемой информации для нормативно-справочной информации;
            $K'_B$~---~коэффициент учета сложности алгоритма ПП и вида используемой информации для баз данных.
            $n_P$~---~количество наборов данных переменной информации;
            $n_{NS}$~---~количество наборов данных нормативно-справочной информации;
            $n_B$~---~количество баз данных.
            Коэффициенты находим по таблице~(группа новизны - В):
            $$K'_P=1.00,$$
            $$K'_{NS}=0.48,$$
            $$K'_B=0.4.$$
            Количество наборов данных, используемых в рамках задачи:
            $$n_P=10,$$
            $$n_{NS}=0,$$
            $$n_B=1.$$
            Находим значение $K_{IA}$:
            $$K_{IA} = \dfrac{1.00*10+0.48*0+0.4*1}{10+0+1}=0.945.$$
            Тогда:
            $$\tau_{RP} = (8+51)*1.00*1.26*0.8*0.8*0.945 = 45.$$

        \subsubsection{Трудоемкость выполнения стадии <<Внедрение>>}
            Расчёт трудоёмкости разработки технического проекта~($\tau_{V}$)~[чел.-дни] производится по формуле:
            $$\tau_{V} = (t^V_{RZ} + t^V_{RP})*K_K*K_R*K_Z,$$
            где $t^V_{RZ}$~---~норма времени, затрачиваемого разработчиком постановки задач на выполнение процедур внедрения программного продукта,~[чел.-дни];
            $t^V_{RP}$~---~норма времени, затрачиваемого разработчиком программного обеспечения на выполнение процедур внедрения программного продукта,~[чел.-дни].
            По таблице принимаем~(функциональное назначение~---~технико-экономическое планирование,
            количество разновидностей форм входной информации~---~1~(файл с текстом программы на исходном языке),
            количество разновидностей форм выходной информации~---~1~(файл с текстом программы на языке Scheme)):
            $$t^V_{RZ} = 9,$$
            $$t^V_{RP} = 11.$$
            Коэффициент $K_K$ и $K_Z$ были найдены выше:
            $$K_K=1.00,$$
            $$K_Z=0.8.$$
            $K_R$~---~коэффициент учета режима обработки информации. По таблице принимаем~(группа новизны~---~В, режим обработки информации~---~реальный масштаб времени):
            $$K_R = 1.26.$$
            Тогда:
            $$\tau_{V} = (9+11)*1.00*1.26*0.8= 21.$$

        Общая трудоёмкость разработки ПП:
        $$\tau_{PP} = 37+77+53+45+21= 233.$$

    \subsection{Расчет количества исполнителей}
    \label{subsec:slaves}
        Средняя численность исполнителей при реализации проекта разработки и внедрения ПО определяется соотношением:
        $$N=\dfrac {t} {F},$$
        где $t$~---~затраты труда на выполнение проекта (разработка и внедрение ПО); $F$~---~фонд рабочего времени.
        Разработка велась 5 месяцев с 1 января 2016 по 31 мая 2016.
        Количество рабочих дней по месяцам приведено в таблице~\ref{tabular:work_day}. Из таблицы получаем, что фонд рабочего времени $$F=96.$$
        \begin{table}[h!]
            \caption{Количество рабочих дней по месяцам\bigskip}
            \centering
            \label{tabular:work_day}
            \begin{tabular}{|c|c|c|}
                \hline
                \bf{Номер месяца} & \bf{Интервал дней}& \bf{Количество рабочих дней} \\ \hline
                1 & 01.01.2016~-~31.01.2016 & 15 \\ \hline
                3 & 01.02.2016~-~29.02.2016 & 20 \\ \hline
                4 & 01.03.2016~-~31.03.2016 & 21 \\ \hline
                5 & 01.04.2016~-~30.04.2016 & 21 \\ \hline
                6 & 01.05.2016~-~31.05.2016 & 19 \\ \hline
                \multicolumn{2}{|c|}{Итого} & 96 \\ \hline
            \end{tabular}
        \end{table}

        Получаем число исполнителей проекта:
        $$N=\dfrac{233}{96}=3$$

        Для реализации проекта потребуются 1 старший инженер и 2 простых инженера.

    \clearpage
    \subsection{Ленточный график выполнения работ}
        На основе рассчитанных в главах \ref{subsec:trud}, \ref{subsec:slaves} трудоёмкости и фонда рабочего времени найдём количество рабочих дней, требуемых для выполнения каждого этапа разработка.
        Результаты приведены в таблице~\ref{tabular:tau_PP}.
        \begin{table}[ht!]
            %\small
            \caption{Трудоёмкость выполнения работы над проектом \bigskip}
            \centering

            \label{tabular:tau_PP}
            \begin{tabular}{|c|c|c|c|c|}
                \hline
                \bf{\specialcell{Номер \\ стадии}} & \bf{Название стадии} & \bf{\specialcell{Трудоёмкость \\ $[$чел.-дни$]$}} & \bf{\specialcell{Удельный вес \\ $[$\%$]$}} & \bf{\specialcell{Количество\\ рабочих дней}} \\ \hline
                1 &  Техническое задание    & 37  & 11  & 10 \\ \hline
                2 & Эскизный проект         & 77  & 24  & 23 \\ \hline
                3 & Технический проект      & 53  & 35  & 34 \\ \hline
                4 & Рабочий проект          & 45  & 25  & 24 \\ \hline
                5 & Внедрение               & 21  & 5   & 5  \\ \hline
                \multicolumn{2}{|c|}{Итого} & 233 & 100 & 96 \\ \hline
            \end{tabular}
        \end{table}

        Планирование и контроль хода выполнения разработки проводится по ленточному графику выполнения работ.
        По данным в таблице~\ref{tabular:tau_PP} в ленточный график (таблица ~\ref{tabular:lenta}), в ячейки столбца “продолжительности рабочих дней” заносятся времена, которые требуются на выполнение соответствующего этапа.
        Все исполнители работают одновременно.
        \begin{table}[ht!]
            \caption{Ленточный график выполнения работ \bigskip}
            \centering
            \label{tabular:lenta}
            \begin{tabular}{|c|c|c|c|c|c|c|c|c|c|c|c|c|c|c|c|c|c|c|c|c|c|c|c|c|}
                \hline

                & & \multicolumn{23}{|c|}{Календарные дни} \\ \cline{3-25}

                \parbox[t]{3mm}{\multirow{4}{*}[2em]{\rotatebox[origin=c]{90}{Номер стадии}}} &
                \parbox[t]{3.6mm}{\multirow{4}{*}[5.8em]{\rotatebox[origin=c]{90}{Продолжительность [раб.-дни]}}} &
                \rotatebox[origin=c]{90}{~01.01.2016~-~03.01.2016~} &
                \rotatebox[origin=c]{90}{~04.01.2016~-~10.01.2016~} &
                \rotatebox[origin=c]{90}{~11.01.2016~-~17.01.2016~} &
                \rotatebox[origin=c]{90}{~18.01.2016~-~24.01.2016~} &
                \rotatebox[origin=c]{90}{~25.01.2016~-~31.01.2016~} &
                \rotatebox[origin=c]{90}{~01.02.2016~-~07.02.2016~} &
                \rotatebox[origin=c]{90}{~08.02.2016~-~14.02.2016~} &
                \rotatebox[origin=c]{90}{~15.02.2016~-~21.02.2016~} &
                \rotatebox[origin=c]{90}{~22.02.2016~-~28.02.2016~} &
                \rotatebox[origin=c]{90}{~29.02.2016~-~06.03.2016~} &
                \rotatebox[origin=c]{90}{~07.03.2016~-~13.03.2016~} &
                \rotatebox[origin=c]{90}{~14.03.2016~-~20.03.2016~} &
                \rotatebox[origin=c]{90}{~21.03.2016~-~27.03.2016~} &
                \rotatebox[origin=c]{90}{~28.03.2016~-~03.04.2016~} &
                \rotatebox[origin=c]{90}{~04.04.2016~-~10.04.2016~} &
                \rotatebox[origin=c]{90}{~11.04.2016~-~17.04.2016~} &
                \rotatebox[origin=c]{90}{~18.04.2016~-~24.04.2016~} &
                \rotatebox[origin=c]{90}{~25.04.2016~-~01.05.2016~} &
                \rotatebox[origin=c]{90}{~02.05.2016~-~08.05.2016~} &
                \rotatebox[origin=c]{90}{~08.05.2016~-~15.05.2016~} &
                \rotatebox[origin=c]{90}{~16.05.2016~-~22.05.2016~} &
                \rotatebox[origin=c]{90}{~23.05.2016~-~29.05.2016~} &
                \rotatebox[origin=c]{90}{~30.05.2016~-~31.05.2016~}
                \\ \cline{3-25}

                & & \multicolumn{23}{|c|}{Количество рабочих дней} \\ \cline{3-25}

                  &    & 0 & 0 & 5 & 5 & 5 & 5 & 5 & 6 & 3 & 5 & 3 & 5 & 5 & 5 & 5 & 5 & 5 & 5 & 3 & 4 & 5 & 5 & 2 \\ \hline
                1 & 10 &   &   & 5 & 5 &   &   &   &   &   &   &   &   &   &   &   &   &   &   &   &   &   &   &   \\ \hline
                2 & 23 &   &   &   &   & 5 & 5 & 5 & 6 & 2 &   &   &   &   &   &   &   &   &   &   &   &   &   &   \\ \hline
                3 & 34 &   &   &   &   &   &   &   &   & 1 & 5 & 3 & 5 & 5 & 5 & 5 & 5 &   &   &   &   &   &   &   \\ \hline
                4 & 24 &   &   &   &   &   &   &   &   &   &   &   &   &   &   &   &   & 5 & 5 & 3 & 4 & 5 & 2 &   \\ \hline
                5 & 5  &   &   &   &   &   &   &   &   &   &   &   &   &   &   &   &   &   &   &   &   &   & 3 & 2 \\ \hline
            \end{tabular}
        %\end{sidewaystable}
        \end{table}

    \subsection{Определение себестоимости программной продукции}
        Затраты, образующие себестоимость продукции (работ, услуг), состоят из затрат на заработную плату исполнителям, затрат на закупку или аренду оборудования, затрат на организацию рабочих мест, и затрат на накладные расходы.

        В таблице~\ref{tabular:zarplata} приведены затраты на заработную плату и отчисления на социальное страхование в пенсионный фонд, фонд занятости и фонд обязательного медицинского страхования (30.5 \%).
        Для старшего инженера предполагается оклад в размере 120000 рублей в месяц, для инженера предполагается оклад в размере 100000  рублей в месяц.
        \begin{table}[ht!]
            %\small
            \caption{Затраты на зарплату и отчисления на социальное страхование \bigskip}
            \centering

            \label{tabular:zarplata}
            \begin{tabular}{|c|c|c|c|c|}
                \hline
                \bf{Должность} &
                \bf{\specialcell{Зарплата \\ в месяц}} &
                %\bf{\specialcell{Кол-во \\ работников}} &
                \bf{\specialcell{Рабочих \\ месяцев}} &
                \bf{\specialcell{Суммарная \\ зарплата}} &
                \bf{\specialcell{Затраты на \\ социальные нужды}} \\ \hline

                Старший инженер & 120000 & 5 & 600000 & 183000 \\ \hline
                Инженер & 100000 & 5 & 500000 & 152500 \\ \hline
                Инженер & 100000 & 5 & 500000 & 152500 \\ \hline
                \multicolumn{3}{|c|}{Суммарные затраты} & \multicolumn{2}{|c|}{2088000} \\ \hline
            \end{tabular}
        \end{table}

        Расходы на материалы, необходимые для разработки программной продукции, указаны в таблице~\ref{tabular:material}.

        \begin{table}[ht!]
            %\small
            \caption{Затраты на материалы \bigskip}
            \centering

            \label{tabular:material}
            \begin{tabular}{|c|c|c|c|c|}
                \hline
                \bf{\specialcell{Наименование \\ материала}} &
                \bf{\specialcell{Единица \\ измерения}} &
                %\bf{\specialcell{Кол-во \\ работников}} &
                \bf{\specialcell{Кол-во}} &
                \bf{\specialcell{Цена за \\ единицу, руб.}} &
                \bf{\specialcell{Сумма, руб.}} \\ \hline

                Бумага А4 & Пачка 400 л. & 2 & 200 & 400 \\ \hline
                \specialcell{Картридж для \\ принтера HP P10025} & Шт. & 3 & 450 & 1350 \\ \hline
                \multicolumn{4}{|c|}{Суммарные затраты} & \multicolumn{1}{|c|}{1750} \\ \hline
            \end{tabular}
        \end{table}

        В работе над проектом используется специальное оборудование~---~персональные электронно-вычислительные машины (ПЭВМ) в количестве 9 шт.
        Стоимость одной ПЭВМ составляет 90000 рублей.
        Месячная норма амортизации K = 2,7\%.
        Тогда за 5 месяцев работы расходы на амортизацию составят $P = 90000 * 3 *  0.027 * 5 = 36450$~рублей.

        Накладные расходы рассчитываются по следующе формуле:
        $$C_n=A_n * C_z$$
        $$N=2.1*1600000=3360000$$

        Общие затраты на разработку программного продукта (ПП) составят

        {\centering{$2088000+1750+36450+3360000=5486200$ рублей.}

        }


    \subsection{Определение стоимости программной продукции}
        Для определения стоимости работ необходимо на основании плановых сроков выполнения работ и численности исполнителей рассчитать общую сумму затрат на разработку программного продукта.
        Если ПП рассматривается и создается как продукция производственно-технического назначения,
        допускающая многократное тиражирование и отчуждение от непосредственных разработчиков, то ее цена~$P$ определяется по формуле:
        $$P = K*C+Pr,$$
        где $C$~---~затраты на разработку ПП (сметная себестоимость);
        $K$~---~коэффициент учёта затрат на изготовление опытного образца ПП как продукции производственно-технического назначения~($K=1.1$);
        $Pr$~---~нормативная прибыль, рассчитываемая по формуле:
        $$Pr= \frac {C * \rho_N} {100},$$
        где $\rho_N$~---~норматив рентабельности, $\rho_N=30\%$;

        Получаем стоимость программного продукта:

        {\centering$P=1.1*5486200 + 5486200*0.3=7680680$~рублей.

        }
    \subsection{Расчет экономической эффективности}
        Основными показателями экономической эффективности является чистый дисконтированный доход~(NPV) и срок окупаемости вложенных средств.
        Чистый дисконтированный доход определяется по формуле:
        $$NPV=\sum_{t=0}^T (R_t-Z_t) * \dfrac{1}{(1+E)^t},$$
        где $T$~---~горизонт расчета по месяцам;
        $t$~---~период расчета;
        $R_t$~---~результат, достигнутый на $t$ шаге (стоимость);
        $Z_t$~---~текущие затраты (на шаге $t$);
        $E$~---~приемлемая для инвестора норма прибыли на вложенный капитал.

        На момент начала 2016 года, ставка рефинансирования 11\% годовых~(ЦБ РФ), что эквивалентно (0.916\% в месяц). В виду особенности разрабатываемого продукта он может быть продан лишь однократно.
        Отсюда получаем $$E=0.00916.$$

        В таблице~\ref{tabular:pdd} находится расчёт чистого дисконтированного дохода. График его изменения приведён на рисунке~\ref{pic:pdd}.

        \begin{table}[ht!]
            %\small
            \caption{Расчёт чистого дисконтированного дохода \bigskip}
            \centering

            \label{tabular:pdd}
            \begin{tabular}{|c|c|c|c|c|}
                \hline
                \bf{\specialcell{Месяц}} &
                \bf{\specialcell{Текущие затраты,\\ руб.}} &
                %\bf{\specialcell{Кол-во \\ работников}} &
                \bf{\specialcell{Затраты с начала \\ года, руб.}} &
                \bf{\specialcell{Текущий доход, \\ руб.}} &
                \bf{\specialcell{ЧДД, руб.}} \\ \hline

                Январь  & 1096890 & 1096890 & 0       & -1096890 \\ \hline
                Февраль & 1096890 & 2193780 & 0       & -2183823.7 \\ \hline
                Март    & 1096890 & 3290670 & 0       & -3260891.4 \\ \hline
                Апрель  & 1096890 & 4387560 & 0       & -4328182.7 \\ \hline
                Мая     & 1098640 & 5486200 & 7680680 & 2019802 \\ \hline

            \end{tabular}
        \end{table}

        \begin{figure}[h!]
            \centering
            \begin{tikzpicture}[scale=1]
                \begin{axis}[ylabel=ЧДД (руб.), xlabel=Количество месяцев с начала проекта,
                ] %\tiny
                    \addplot coordinates {
                        (1, -1096890)
                        (2, -2183823.7)
                        (3, -3260891.4)
                        (4, -4328182.7)
                        (5, 2019802)
                    };
                \end{axis}
            \end{tikzpicture}
            \caption{График изменения чистого дисконтированного дохода}
            \label{pic:pdd}
        \end{figure}

        Согласно проведенным расчетам, проект является рентабельным.
        Разрабатываемый проект позволит превысить показатели качества существующих систем и сможет их заменить.
        Итоговый ЧДД составил: $2019802$ рубля.

    \subsection{Результаты}
        В рамках организационно-экономической части был спланирован календарный график проведения работ по созданию подсистемы поддержки проведения диагностики промышленных, а также были проведены расчеты по трудозатратам.
        Были исследованы и рассчитаны следующие статьи затрат: материальные затраты; заработная плата исполнителей; отчисления на социальное страхование; накладные расходы.

        В результате расчетов было получено общее время выполнения проекта, которое составило $96$ рабочих дней, получены данные по суммарным затратам на создание и разработку функционального языка, которые составили $5486200$ рублей.
        Согласно проведенным расчетам, проект является рентабельным.
        Цена данного программного проекта составила $7680680$ рублей, итоговый ЧДД составил $2019802$ рублей.
