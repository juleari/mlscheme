\documentclass[14pt]{extarticle}

\usepackage[14]{extsizes}
\usepackage{ucs}
\usepackage{amsmath}            % gather* для вывода формул по центру страницы
\usepackage[utf8x]{inputenc}    % Включаем поддержку UTF8
\usepackage[russian]{babel}     % Включаем пакет для поддержки русского языка

\usepackage[
    left=2cm,           % Поле левое : 200 мм
    right=2cm,          % Поле правое : 200 мм
    top=2cm,            % Поле верхнее: 200 мм
    bottom=2cm,         % Поле нижнее : 200 мм
    bindingoffset=0cm]{geometry}

\setlength{\parindent}{1.25cm}          % Абзацный отступ: 1,25 см
\usepackage{indentfirst}                % 1-й абзац имеет отступ

\usepackage[pdftex]{graphicx, color}
\usepackage{color}
\usepackage{tikz}
\usepackage{url}                % использование URL в библиографии
\usepackage{listings}           % использование листингов кода
\usepackage[nooneline]{caption}
\captionsetup[table]{justification=raggedleft}
\captionsetup[figure]{justification=centering,labelsep=endash}
%\usepackage{array}

\usepackage[nodisplayskipstretch]{setspace}
\setstretch{1.5}

\usepackage{caption}
\usepackage{graphicx}
\usepackage{subcaption}
\usepackage{cases}
\usepackage{placeins}
\usepackage{longtable} 
\usepackage{multirow}
\usepackage{hhline}

\renewcommand{\baselinestretch}{1.5}

\definecolor{grey}{HTML}{EEEEEE}
% вставка листингов с кодом
\lstset{inputencoding=utf8x,
        extendedchars=false,
        keepspaces=true,
        basicstyle=\small\sffamily,
        numbers=left,
        numberstyle=\tiny,
        numbersep=5pt,
        stepnumber=1,                % размер шага между двумя номерами строк
        backgroundcolor=\color{white},
        showspaces=false,            % показывать или нет пробелы специальными отступами
        showstringspaces=false,      % показывать или нет пробелы в строках
        showtabs=false,              % показывать или нет табуляцию в строках
        frame=none,
        tabsize=2,                   % размер табуляции по умолчанию равен 2 пробелам
        captionpos=t,                % позиция заголовка вверху [t] или внизу [b] 
        breaklines=true,             % автоматически переносить строки (да\нет)
        breakatwhitespace=false}     % переносить строки только если есть пробел
        
%\input{listings-python.prf}
\renewcommand{\lstlistingname}{Листинг}

\setcounter{tocdepth}{4}        % chapter, section, subsection, subsubsection и paragraph
\setcounter{secnumdepth}{4}

\parindent=1,25cm               % красная строка = 1 см
\usepackage{enumitem}
\setlist[enumerate,1]{leftmargin=2.25cm}
\setlist[itemize]{leftmargin=2.25cm}
\graphicspath{{pics/}}
\DeclareGraphicsExtensions{{.jpg}}
\usepackage{color, colortbl}


\usepackage[pdftex]{graphicx}  % поддержка картинок для пдф
\usepackage{rotating}
\usepackage{graphicx}

\usepackage{tikz} %для рисования графиков
\usepackage{pgfplots}
\usepackage{rotating}
\newcommand*\rot{\rotatebox{90}}

\usepackage{ccaption}%изменяет подпись к рисунку
\makeatletter 
\renewcommand{\fnum@figure}[1]{Рисунок~\thefigure~---~\sffamily}
\makeatother

\newcommand{\specialcell}[2][c]{%
   \begin{tabular}[#1]{@{}c@{}}#2\end{tabular}}

\begin{document}
    \pgfplotsset{compat=1.8}

\thispagestyle{empty}
\newpage
{
\centering


\textbf{
МОСКОВСКИЙ ГОСУДАРСТВЕННЫЙ ТЕХНИЧЕСКИЙ УНИВЕРСИТЕТ ИМЕНИ Н. Э. БАУМАНА \\
Факультет информатики и систем управления \\
Кафедра теоретической информатики и компьютерных технологий}

\vfill

\hfill\parbox{7cm} {
УТВЕРЖДАЮ:\\
Заведующий кафедрой ИУ-9 \hfill \\
\underline{\hspace{4cm}}(И.П. Иванов)\hfill \\
<<\underline{\hspace{0.5cm}}>>\underline{\hspace{3cm}}201\underline{\hspace{0.5cm}}г.\hfill \\
}

\bigskip
\bigskip
\bigskip
\bigskip
\bigskip
\bigskip
\bigskip
\bigskip

\vfill

{\large\bf Обзор} \\
к дипломному проекту \\
<<Функциональный язык программирования с динамической типизацией и ML-подобным синтаксисом>>

\vfill

\hfill\parbox{7cm} {
Исполнитель: Ю.А. Волкова \\
Группа: ИУ9-111
}

\bigskip
\bigskip
\bigskip
\bigskip
\bigskip
\bigskip
\bigskip

\vfill

Руководитель \\
квалификационной работы \\
А.В. Дубанов

\vspace{\fill}
}

\clearpage

    \tableofcontents
    \clearpage
    % Введение, актуальность -- 1
\section*{Введение}
\addcontentsline{toc}{section}{Введение}
    Ранее, язык программирования был воплощением какой-либо парадигмы, концепции.
    Сейчас --- при разработке языка стремятся добиться удобства.
    Особое внимание уделяется краткости, удобству написания и чтения кода.

    К настоящему времени разработано несколько тысяч языков программирования, из которых около двадцати являются широко распространёнными~\cite{TIOBE}.
    У всех есть свои сильные стороны и под конкретные задачи выбирается конкретный язык программирования, наиболее подходящий для этих целей.
    Так, например, для написания кросс-платформенных приложений используются Java или C++~\cite{p_c_lisp}.
    Для браузерных расширений и сайтов --- JavaScript.

    Однако, для большинства языков, выигрыш от их выбора не так очевиден.
    Пользователи языка Python считают код на своём языке простым для понимания, потому что чтение программы на Python напоминает чтение текста на английском языке.
    Это позволяет сосредоточится на решении задачи, а не на самом языке.
    Lisp является программируемым языком программирования, что позволяет изменять и дополнять его под конкретные задачи.
    Язык OCaml благодаря системе вывода типов позволяет писать высокоэффективные и безопасные приложения.

    Анализ текущего состояния в разработке языков программирования показал, что существует необходимость в языке программирования, ориентированном на быструю разработку сценариев (скриптов), первоначальном обучении программированию и исследовательском программировании (когда изначально неизвестно как программа должна работать).

    Для этих целей, на наш взгляд, должен существовать функциональный язык с <<дружелюбным>> синтаксисом, который подразумевает инфиксную нотацию в записи арифметических выражений.
 %1
    \clearpage
    % Языки-прототипы --- 7
\section{Языки-прототипы}
    \subsection{Семейство языков Lisp, язык Scheme}
        $Lisp$ (LISt Processing language) --- функциональный язык программирования с динамической типизацией. 
        Он был создан для символьной обработки данных~\cite{p_c_lisp}.
        $Scheme$ --- диалект Lisp'а, использующий хвостовую рекурсию и статические области видимости переменных~\cite{scheme_doc}.
        Scheme --- высокоуровевый язык общего назначения, поддерживающие операции над такими структурами данных как строки, списки и векторы.

        Язык Scheme был создан как учебный, однако, в последнее время используется для написания текстовых редакторов, оптимизирующих компиляторов, операционных систем, графических пакетов и экспертных систем, вычислительных приложений и систем виртуальной реальности.
        Язык оказал сильное влияние на многие современные языки программирования, такие как Haskell (и языки семейства ML в целом), Rust, Python, JavaScript.

        В Scheme реализована \textit{сборка мусора} --- одна из форм управления памятью, удаляющая оттуда объекты, которые уже не будут востребованы.
        
        Все объекты, в том числе функции, являются \textit{объектами первого класса}, что позволяет присваивать функции переменным, возвращать функции и принимать их в качестве аргументов.
        Все переменные и ключевые слова объедены в области видимости, а программы на языке имеют блочную структуру.
        
        Как и во многих других языках программирования процедуры на языке Scheme могут быть рекурсивными.
        Все хвостовые рекурсии оптимизируются.

        Scheme поддерживает определение произвольных структур с помощью \textit{продолжений}.
        Продолжения сохраняют текущее состояние программы и позволяют продолжить выполнение с этого момента из любой точки программы.
        Этот механизм удобен для перебора с возвратом, многопоточности и сопрограмм~\cite{scheme_pl}.

        Scheme также позволяет создавать синтаксические расширения для написания процедур трансформации новых синтаксических форм в существующие~\cite{scheme_pl}.

        Функции на языке Scheme могут иметь произвольное число аргументов.
        Тем переменным, наличие которых необходимо, присваиваются имена, а остальные можно получить из списка оставшихся.

        В Scheme есть \textit{статические переменные} --- долговременные переменные, существующие на протяжении функции.
        Они отличаются от глобальных переменных тем, что существуют только внутри функции, могут хранить свои значения между вызовами, но при этом не доступы извне.
        Это позволяет организовывать \textit{мемоизацию} (сохранение результатов выполнения функций для предотвращения повторных вычислений).
        Это является одним из способов оптимизации.
        При использовании оптимизации перед началом вычислений проверяется вызывалась ли ранее эта функция с этим набором аргументов.
        Если не вызывалась, то результат вычисляется, сохраняется в статической переменной и возвращается.
        Если функция уже вызывалась с данным набором аргументов, то возвращается сохранённый ранее результат.
        
        Язык Scheme обладает полноценными средствами символьной обработки.

        Как и в большинстве языков семейства Lisp, в языке Scheme как код, так и данные представлены в виде упорядоченных последовательностей значений --- списков.
        Таким образом, способ представления кода не отличается от способа представления данных.
        Это позволяет работать с кодом как с данными, в том числе разрабатывать самомодифицирующиеся программы.
        
        В Scheme используется префиксная нотация.
        Выражения являются списками с оператором во главе этого списка.
        Такой способ записи позволяет, например, сложить сразу несколько значений.
        Но выражения с несколькими операторами трудны для восприятия из-за большого количества скобок.
        Это требует от разработчика использовать специализированные текстовые редакторы, поддерживающие контроль парности скобок и принятый стиль форматирования кода.
        Таких редакторов немного, и они либо недружелюбны к начинающему программисту (Emacs, vim), либо ориентированы на определённый диалект языка (Racket).
        \clearpage
        Пример функции вычисления факториала числа, написанный на языке Scheme:

        \lstinputlisting{examples/scheme_factorial}

    \subsection{Язык Prolog}
        $Prolog$ (PROgramming in LOGic) --- язык, объединяющий в себе логическое и алгоритмическое программирование.
        Этот язык специально разрабатывался для систем обработки естественных языков, исследований искуственного интеллекта и экспертных систем.
        В настоящее время этот язык применяется не очень широко~\cite{TIOBE}, хотя современный Prolog во многом является языком программирования общего назначения.

        Язык использует \textit{сопоставление с образцом} (в том числе повторное использование).
        Сопоставление с образцом в языке Prolog базируется на унификации, в частности используя алгоритм Мартелли-Монтэнери~\cite{unify}.
        Этот алгоритм сопоставляет значения двух элементов выражения, находящихся на одинаковой позиции и в случае успеха приравнивает значения на других позициях.

        Ключевыми особенностями языка Prolog являются унификация и перебор с возвратом.
        Унификация показывает как две произвольные структуры могут быть равными.
        Процессоры Prolog'а используют стратегию поиска, которая пытается найти решение проблемы перебором с возвратом по всем возможным путям, пока один из них не приведёт к решению~\cite{prolog}.
        Программа на языке Prolog представляет собой набор фактов и правил.

        Входной точкой в программу на языке Prolog является запрос.
        Ответом на него может быть либо список подошедших шаблонов, либо \verb!false!, означающий, что совпадений не найдено.
        \clearpage
        Рассмотрим пример программы на языке Prolog.

        \lstinputlisting{examples/prolog_def}

        Эта программа состоит из трёх предложений, каждое из которых объявляет один факт об отношении \verb$street$.
        Например, факт \verb$street( moscow, arbat )$ описывает, что улица \verb$arbat$ относится к \verb$moscow$.
        После передачи соответствующей программы в систему Prolog последней можно задать некоторые вопросы об отношении \verb$street$.
        Например, можно узнать все города, в которых есть улица \verb$arbat$.
        Сделать это можно введя в терминал следующий запрос:

        \lstinputlisting{examples/prolog_call}

        После этого Prolog начнёт отыскивать все пары город-улица, где улица --- \verb$arbat$.
        Решения отображаются на дисплее по одному до тех пор, пока Prolog получает указание найти следующее решение (в виде точки с запятой) или пока не будут найдены все решения~\cite{prolog_bratko}.
        Ответы выводятся следующим образом:

        \lstinputlisting{examples/prolog_res}

    \subsection{Семейство языков ML, Язык OCaml}
        $ML$ (Meta Language) --- семейство языков с полиморфным выводом типов и обработкой исключений.
        Языки ML не являются чистыми функциональными языками.
        
        $OCaml$ --- самый распространённый в практической работе диалект ML.

        Он включает в себя сборку мусора для автоматического управления памятью.
        Как и в Lisp, функции являются объектами первого класса.
        Это значит, что функции можно передавать в качестве аргумента, возвращать в качестве значения и присваивать их переменным.

        OCaml использует статическую проверкую типов, что позволяет увеличить скорость сократить количество ошибок во время исполнения.
        При этом поддерживается \textit{параметрический полиморфизм} --- свойство семантики системы типов, позволяющее обрабатывать данные разных типов идентичным образом.
        Это даёт возможность создавать абстракции и работать с разными типами данных.

        Механизм автоматического вывода типов позволяет избегать тщательного определения типа каждой переменной, вычисляя его по тому как она используется.
        Кроме того, в OCaml, как и в Scheme, есть поддержка неизменяемых данных.

        Также, поддерживаются \textit{алгебраические типы данных} (составные типы, имеющие набор конструкторов, каждый из которых принимает на вход значения определённых типов и возвращает значение конструируемого типа) и сопоставление с образцом, чтобы определять и управлять сложными структурами данных~\cite{ocaml}.

        Пример вычисления факториала числа на языке OCaml, используя сопоставление с образцом:

        \lstinputlisting{examples/ocaml_factorial}

    \subsection{Язык Haskell}
        $Haskell$ --- чистый функциональный язык программирования общего назначения.
        \textit{Чистота} означает, что результат вычислений, производимых функциями не зависит от состояния программы.
        Он зависит только от набора входных параметров, а значит, уже вычисленные значения не изменятся.
        А значит, при следующем вызове функции с таким же набором фактических аргументов это значение можно уже не вычислять.

        Кроме того, Haskell является ленивым, что позволяет не вычислять значения, пока это не нужно.
        \textit{Ленивость} ускоряет работу программы, однако позволяет допускать ошибки, так как выражение, содержащее ошибку может быть невыполнено.

        Haskell обладает статической сильной полной типизацией с атоматическим выводом типов.
        
        Также, Haskell поддерживает функции высшего порядка и частичное применение (\textit{каррирование} --- преобразование функции от многих аргументов в набор функций, каждая из которых принимает один аргумент).

        Haskell ограниченно используется для написания сценариев.
        Дело в том, что Haskell удлиняет код для коротких скриптов за счёт статической типизации и необходимости использования монад, для организации императивного кода и функций с побочными эффектами.
        Это приводит также к усложнению кода, к введению лишних сущностей.
        Монады и необходимость следить за типами означают, что у языка высокий порог вхождения, что делает его неподходящим для первоначального обучения.
        Статическая проверка типов сокращает время выполнения программы, однако увеличивает затраты времени на проверку типов при загрузке скрипта интерпретатору.
        Это также делает Haskell менее удобным для написания скриптов, по скравнению с языками с динамической типизацией.

        Haskell --- краткий и элегантный язык~\cite{haskell}.
        В нём используется минимум ключевых слов.
        Например, для определения функции во многих языках необходимо использование ключевых слов (например, \verb!define! в Scheme или \verb!function! в OCaml).
        Нотация в целом похожа на математическую.
        Также, как в Python вместо скобок, ограничивающих блоки кода (Scheme, C, JavaScript) и разделительных знаков (C, OCaml) в нём используются отступы.
        Отступы в Haskell являются \textit{синтаксическим сахаром} (синтаксической возможностью, не влияющей на выполнение программы, но упрощающей написание кода).
        Фактически препроцессор компилятора заменяет их на ограничивающий блок (\verb${}$) и разделители (\verb$;$).
        Это делает код легче для восприятия, так как сразу видна вложенность.
        Однако, в отличие от Python, где отступы обязаны быть одинаковыми, в Haskell важно лишь, чтобы отступ вложенной операции был больше родительской.
        \clearpage
        Пример вычисления факториала числа на языке Haskell:

        \lstinputlisting{examples/haskell_factorial.hs}

    \subsection{Концепция языка}
        %Традиционно скрипт --- это текст, а не байт-код или нативный код, даже если скрипт компилируется.
        Язык разрабатывается для написания скриптов, первоначального обучения программированию и исследовательского программирования.

        Для этих целей больше подходит динамическая типизация, как в Scheme для большей гибкости и краткости кода.

        Чтобы снизить порог вхождения Язык должен быть содержать минимум ключевых слов и разделяющих знаков, как в Haskell.

        Для лучшего понимания программы запись функций и результат их вычислений должен быть приближен к математической.
        Для работы с математическими формулами необходимы символьные вычисления (Scheme, Prolog).

        Короткие программы легче поддерживать.
        И, как правило, в них меньше возможности допустить ошибку.
        Для краткости кода в нём должно быть реализовано сопоставление с образцом как в OCaml и повторное использование переменных как в Prolog.

        Для оптимизации рекурсивных функций (таких как функция вычисления факториала и чисел фибоначчи) необходима возможность мемоизации функций (как в языке Haskell).

        Функция вычисления факториала числа на Языке должна выглядеть следующим образом:

        \lstinputlisting{examples/factorial}

    \subsection{Выбор языка реализации и целевого языка для компилятора}
        Для реализации динамической типизации удобнее всего выбрать язык с таким типом типизации.
        Таковым является язык Scheme.

        Символьные вычисления также реализованы в Scheme.

        Для реализации была выбрана спецификация языка R5RS, так как она поддерживается всеми известными компиляторами (GNU Guile, Chicken Scheme, Racket).
 %3
    \clearpage
    % Синтаксис ФЯП -- 5
\section{Синтаксис функционального языка программирования}
%\addcontentsline{toc}{section}{Синтаксис функционального языка программирования}
    Для описания синтаксиса языка используются расширенная форма Бэкуса-Наура (РБНФ). 
    Альтернатива обозначается символом '|'. 
    '*' после выражения означает, что оно может быть включено $0$ и более раз, '+' - $1$ и более, '?' - $0$ или $1$ раз.\cite{skor}
    Нетерминальные символы начинаются с заглавной буквы. 
    Терминальные либо начинаются малой буквой, либо состоят целиком из заглавных букв.
    Правила записываются в виде \verb!Идентификатор ::= выражение!,
    где идентификатор -- нетерминал, а выражение -- соответствующая правилам РБНФ комбинация терминальных и нетерминальных символов и специальных знаков.
    Точка в конце — специальный символ, указывающий на завершение правила.

    Язык является регистрозависимым.

    \subsection{Словарь и представление}
        Существуют следующие виды токенов: идентификатор, число, строка, операторы и ключевые слова.
        Пробельные символы игнорируются, если они не существенны для разделения двух последовательных токенов.

        \begin{itemize}
            \item Число -- целочисленная, вещественная константа, дробь или комплексное число.
            Число (в том числе дробь и вещественное число) может быть записано в шестнадцатеричной системе счисления.
            
            Примеры чисел: \verb!#xA9.F!, \verb!4/7!, \verb#2+7i#, \verb!#x7/ad!.
            \lstinputlisting{syntax/number}
            \item Идентификатор -- последовательность символов, исключающая пробельные символы, точки, скобки, кавычки, двоеточие.
            В отличие от большинства языков идентификаторы могут начинаться с цифр.
            В этом случае идентификатор целиком не должен являться числом.

            Примеры идентификаторов: \verb#day-of-week#, \verb#0->1#, \verb#nil?#, \verb#%2!0?#.
            \lstinputlisting{syntax/ident}
            \item Строки -- последовательности символов, заключённые в двойные(") кавычки.
            
            Примеры строк: \verb#"valid string"#, \verb#"it's a beautiful day"#.
            \item Булевые константы -- константы, отвечающие за истинное и ложное значение.
            
            Истина: \verb!#t!. Ложь: \verb!#f!.
            \lstinputlisting{syntax/bool}
            \item Операторы и ключевые слова -- специальные символы, пары символов и слова, зарезервированные системой.

            Список таких символов: 
            
            \verb,( ) [ ] { } . : \ < <= > >= ! != =,

            \verb,- + ++ / // % * ** | || && ^ <- ->,

            Список таких слов: \verb!scheme!, \verb!mod!, \verb!if!, \verb!zero?!, \verb!eval!, \verb!abs!, \verb!odd?!, \verb!even?!, \verb!div!, \verb!round!, \verb!reverse!,
             \verb!null?!, \verb!not!, \verb!sin!, \verb!cos!, \verb!tg!, \verb!ctg!, \verb!eq?!, \verb!eqv?!, \verb!equal?!, \verb!gcd!, \verb!lcm!, \verb!expt!, \verb!sqrt!.
        \end{itemize}

    \subsection{Объявление функций}
        Функции делятся на именованные и безымянные $\lambda$-функции.
        Объявление именованных функций (FunctionDefinition) состоит из имени функции, её аргументов, знака присваивания $\leftarrow$ и тела функции.
        Аргументами функции при объявлении могут быть 
        \begin{itemize}
            \item идентификаторы, не являющиеся ключевыми словами (SimpleIdent);
            \item числа (Number);
            \item списки, не содержащие выражений (ListDeclaration);
            \item и <<продолжение>> (ContinuousDeclaration) -- этот элемент может быть в списке аргументов только последним и он заменяет все аргументы, которые могут быть переданы в эту функцию после уже объявленных.
        \end{itemize}
        
        \lstinputlisting{syntax/func_def}

        Объявление безымянных функций (LambdaFunction) похоже на объявление именных.
        Вместо идентификатора, отвечающего за имя функции, используется символ \verb,\,.

        \lstinputlisting{syntax/lambda}

    \subsection{Тело функции}
        Тело функции, также может содержать в себе определения функций.
        Тело функции обязательно содержит выражение.
        Выражений может быть несколько, но вернёт функция только последнее выражение.

        Все выражения и определения внутри тела функции должны иметь больший отступ, чем имя этой функции.

        \lstinputlisting{syntax/func_body}

    \subsection{Выражение}
        Выражение (Expression) может быть
        \begin{itemize}
            \item условным выражением (IfExpression);
            \item безымянной функцией (LambdaFunction);
            \item математическим выражением (MathExpression).
        \end{itemize}

        \lstinputlisting{syntax/expr}

    \subsection{Условный оператор}
        Условный оператор $if$ похож на оператор $cond$ в $Scheme$.
        Структура его аргументов устроена следующим образом:
        \begin{enumerate}
            \item после вертикальной черты следует условие.
                В случае если условие отсутствует, считается, что условие истинно.
                Таким образом, опустив условие можно организовать ветку $else$ для привычных условных операторов
                или $default$ для $switch-case$ оператора;
            \item после стрелочки следует выражение, которое выполнится, если условие перед стрелочкой было выполнено.
                Если условие было выполнено, то после вычисления выражения действие оператора заканчивается и остальные условия не проверяются.
                Если условие выполнено не было -- переходим к проверке следующего условия.
        \end{enumerate}

        \lstinputlisting{syntax/if}

        Пример использования конструкции $if$:

        \lstinputlisting{examples/sign-if}

    \clearpage
    \subsection{Математическое выражение}
        Математическое выражение может быть
        \begin{itemize}
            \item строкой или конкатинацией строк (StringExpression);
            \item вызовом функции (FunctionCall) или применением функции (FunctionApply);
            \item списком или конкатинацией списков (ListExpression).
                Элементы списка при этом не могут содержать вызов функции.
                Поэтому, если это необходимо, то нужно использовать применение функции к списку аргументов.
            \item объединение математических выражений с помощью операторов и скобок.
        \end{itemize}
        
        \lstinputlisting{syntax/math_expr}
 %5
    \clearpage
    % Способы реализации ЯП (что-то там с DSL Clojure?) -- 5
\section{Реализация языка программирования}
    \subsection{Компилятор}
    \clearpage
    % Компоненты среды разработки --- 7
\section{Компоненты среды разработки}
    Компилятор языка состоит из пяти основных компонентов.
    К ним относятся: чтение входного потока, лексический, синтаксический и семантический анализаторы и генератор кода.
    \subsection{Чтение входного потока}
    На этом этапе происходит чтение текста программы из файла и предварительное его разделение на <<слова>>.

    <<Словом>> является:
    \begin{itemize}
        \item любая последовательность символов, заключённая в двойные кавычки;
        \item любой терминальный символ (символ переноса строки, пробел, табуляция, окончание файла);
        \item последовательность символов, образующая слово $scheme$, если она не входит в состав другого <<слова>>;
        \item последовательность символов, заключённых в круглые скобки, следующая за словом $scheme$ и отделённая от него одним или несколькими <<словами>>, являющимися терминальными символами;
        \item любая иная последовательность символов, не содержащая терминальных символов.
    \end{itemize}

    После этого этапа мы получаем список <<слов>>, который подаётся на вход лексическому анализатору.
    
    \subsection{Лексический анализ}
    На этом этапе осуществляется преобразование последовательности <<слов>> в последовательность токенов.
    Это происходит по следующей схеме.

    Сначала проверяется, входит ли данное <<слово>> в список ключевых слов.
    Если проверка оказывается успешной, то вычисляются координаты и создаётся токен с тэгом \verb$tag-kw$, значением которого будет являтся само <<слово>>.
    При этом, если ключевое слово является словом $scheme$, то ближайшее следующее за ним нетерминальное слово будет значением нового токена с тэгом \verb$tag-schm$.

    Затем осуществляется проверка является ли <<слово>> корректным числом.
    Если так, то высчитываются его координаты и создаётся токен с числовым тэгом.
    Значению этого токена присваивается вычисленное на этапе проверки число.

    После этого проверяется, является ли <<слово>> <<строкой>>, то есть последовательностью символов, заключённых в кавычки.
    Аналогично предыдущим пунктам вычисляются координаты и создаётся новый токен с тэгом \verb$tag-str$.

    В случае, если ни одна из этих проверок не увенчалась успехом, данное <<слово>> преобразуется в последовательность символов и дальнейшая проверка будет по-символьной.

    Если среди последовательности символов нам встретился символ $;$, дальнейшие символы этого <<слова>> и последующих игнорируются, пока не встретится символ переноса строки.

    Если среди последовательности символов встречается один из следующих: \verb!( ) [ ] { } . :!, то эта последовательность разбивается на три части:
    \begin{enumerate}
        \item[1)] собственно символ;
        \item[2)] последовательность символов до него (возможно, пустая);
        \item[3)] последовательность символов после (возможно, пустая).
    \end{enumerate}
    Первая часть преобразуется в токен, с тэгом, соответствующем символу.
    Вторая и третья подвергаются анализу как отдельные <<слова>>.

    Для каждого элемента группы символов \verb,\ < ^ # !,, существует свой тэг.
    Но он останется у токена, только если этот символ был в <<слове>> единственным.

    Итоговый тэг для следующей группы символов: \verb!- + * t f / > | & =! определяется по тому является ли символ первым в слове или он следует за каким-то другим.
    Например, символ $-$ может получить тэг \verb$tag-mns$, если этот символ в слове первый.
    Если перед ним в <<слове>> стоит символ $<$ с тэгом \verb$tag-lwr$, то итоговый тэг изменится на \verb$tag-to$.
    Но если перед символом $-$ стоит какой-то другой символ или последовательность, то итоговый тэг будет --- \verb$tag-sym$, и вся эта последовательность станет токеном идентификатора.
    
    \subsection{Синтаксический анализ}
    Синтаксический анализ осуществляется с помощью метода рекурсивного спуска.
    При этом разбор математических выражений осуществляется с помощью алгоритма сортировочной станции.
        \subsubsection{Метод рекурсивного спуска}
        Для каждого нетерминала, описанного в пункте 2 создаётся функция.
        В этой функции сначала определяется вложенность токена по отступу.
        Затем проверяется выполнение правила грамматики, определяющего нетерминал\cite{skor}.

        Текущий токен, хранится в глобальной для функций-правил переменной.
        Входная функции соответствует правилу $Program$.

        При возникновении ошибки, она записывается и список ошибок и продолжается проверка.
        После обработки последнего токена печатается печатается список ошибок с указанием координат ошибки и сообщением о её типе.

        \subsubsection{Алгоритм сортировочной станции}
        Для будущей возможности определения функций-операторов с приоритетами операторов для разбора выражений был выбран алгоритм сортировочной станции.
        В случае корректного математического выражения мы получим его запись в обратной польской нотации.
        Выражения на Scheme имеют прямую польскую запись.
        Поэтому для их вычисления, понадобится только преобразовать обратную польскую запись к прямой.

    \subsubsection{Дерево разбора}
    Синтаксический анализатор возвращает дерево разбора.
    Для функции вычисления факториала, описанной в пункте 1, и её вызовов 
    \\ \verb,n! 5,
    \\ \verb,n! 10
    \\ будет построено следующее дерево:

    \lstinputlisting{examples/ast_factorial}
    \subsection{Семантический анализ}
    Семантический анализатор принимает синтаксическое дерево, созданное в процессе синтаксического анализа.
    Семантический анализатор составляет модель программы, которая состоит из таблицы символов и списка выражений.

    Проходя по синтаксическому дереву семантический анализатор составляет таблицу символов --- отображение идентификаторов символов, в описания соответствующих этим символам сущностей\cite{skor}.
    Здесь символ --- именованная сущность, определяемая парой \verb!<name, info>!, где \verb,name, --- идентификатор сущности, а \verb,info, --- её описание.

    Описание символа представляет собой список возможных значений этого символа.
    Одно такое значение хранится в виде вектора из трёх элементов:
    \begin{enumerate}
        \item описания аргументов;
        \item списка внутренних определений;
        \item списка выражений.
    \end{enumerate}

    Описание аргументов также является вектором, но состоящим из четырёх элементов:
    \begin{enumerate}
        \item функции проверки количества аргументов;
        \item списка функций проверки значений аргументов;
        \item списка имён аргументов;
        \item функции проверки равенства аргументов с одинаковыми идентификаторами.
    \end{enumerate}
    Все эти функции создаются в процессе семантического анализа.

    Повторяющиеся имена заменяются на те, которые не могут быть идентификаторами Языка, но являются корректными в Scheme.
    Для замены создаётся имя, начинающееся с двоеточия.
    Таким же образом заменяются цифры.
    Для списков хранится список имён.
    <<Продолжения>> записываются в виде \verb,(:continuous xs), где \verb/xs/ -- имя переменной, в которой будут хранится все неописанные в основном списке аргументы.

    Функции проверки количества аргументов отличаются для функций, у которых в списке аргументов есть <<продолжения>> и нет.
    Для первых --- это проверка на то, что аргументов не меньше, чем перечислено до <<продолжения>>.
    Для вторых --- проверка количества на равенство.

    Список внутренних определений является таблицей символов.

    Список выражений является промежуточным вариантом их представления между синтаксическим деревом и итоговым сгенерированным кодом.
    Выполняется преобразование вызовов функций и применения функций в удобный для итоговой генерации вид.

    Когда встречается вызов функции осуществляется проверка существования соответствующего символа в таблице и выполняются проверки аргументов.
    В случае если такой символ не найден или если список аргументов не соответствует ни одному из описания --- в список ошибок семантического анализатора добавляется ошибка с описанием.

    Для синтаксического дерева, приведённого в предыдущем пункте получена следующая модель программы:
    
    \lstinputlisting{examples/model_factorial}
    
    \subsection{Генерация кода}
    На этапе генерации кода выполняется преобразование математических выражений из обратной польской нотации в прямую
    Для всех символов из таблицы генерируются функции.
    Объединение возможных значений функции осуществлено с помощью видоизменённой $cond$-конструкции.
    В случае выполнения условий из описания аргументов осуществляется вычисление списка внутренних определений и выражений.
    Иначе проверяются условия из описания аргументов других возможных значений.
    На случай, если ни одна из проверок не увенчается успехом генерируется условие с выводом ошибки.

    Сгенерированный код приписывается к базовым функциям.

    Пример кода, сгенерированного на основе модели из предыдущего пункта:

    \lstinputlisting{examples/gen_factorial}

    \clearpage
    \section{Технико-экономическое обоснование}
    Разработка программного обеспечения~---~достаточно трудоемкий и длительный процесс, требующий выполнения большого числа разнообразных операций.
    Организация и планирование процесса разработки программного продукта или программного комплекса при традиционном методе планирования предусматривает выполнение следующих работ:
    \begin{itemize}
        \item формирование состава выполняемых работ и группировка их по стадиям разработки;
        \item расчет трудоемкости выполнения работ;
        \item установление профессионального состава и расчет количества исполнителей;
        \item определение продолжительности выполнения отдельных этапов разработки;
        \item построение календарного графика выполнения разработки;
        \item контроль выполнения календарного графика.
    \end{itemize}

    Трудоемкость разработки программной продукции зависит от ряда факторов, основными из которых являются следующие: степень новизны разрабатываемого программного комплекса, сложность алгоритма его функционирования, объем используемой информации, вид ее представления и способ обработки, а также уровень используемого алгоритмического языка программирования.
    Чем выше уровень языка, тем трудоемкость меньше.

    По степени новизны разрабатываемый проект относится к \textit{группе новизны В} – разработка программной продукции, имеющей аналоги.

    По степени сложности алгоритма функционирования проект относится к \textit{3 группе сложности} - программная продукция, реализующая алгоритмы стандартных методов решения задач.

    По виду представления исходной информации и способа ее контроля программный продукт относится к \textit{группе 12} - исходная информация представлена в форме документов, имеющих различный формат и структуру и \textit{группе 22} - требуется печать документов одинаковой формы и содержания, вывод массивов данных на машинные носители.

    \subsection{Трудоемкость разработки программной продукции}
    \label{subsec:trud}
        Трудоемкость разработки программной продукции~($\tau_{PP}$) может быть определена как сумма величин трудоемкости выполнения отдельных стадий разработки программного продукта из выражения:
        $$\tau_{PP} = \tau_{TZ} + \tau_{EP} + \tau_{TP} + \tau_{RP} + \tau_{V},$$
        где $\tau_{TZ}$~---~трудоемкость разработки технического задания на создание программного продукта;
        $\tau_{EP}$~---~трудоемкость разработки эскизного проекта программного продукта;
        $\tau_{TP}$~---~трудоемкость разработки технического проекта программного продукта;
        $\tau_{RP}$~---~трудоемкость разработки рабочего проекта программного продукта;
        $\tau_{V}$~---~трудоемкость внедрения разработанного программного продукта.

        \subsubsection{Трудоемкость разработки технического задания}
            Расчёт трудоёмкости разработки технического задания~($\tau_{TZ}$)~[чел.-дни] производится по формуле:
            $$\tau_{TZ} = T^Z_{RZ} + T^Z_{RP},$$
            где $T^Z_{RZ}$~---~затраты времени разработчика постановки задачи на разработку ТЗ,~[чел.-дни];
            $T^Z_{RP}$~---~затраты времени разработчика программного обеспечения на разработку ТЗ,~[чел.-дни].
            Их значения рассчитываются по формулам:
            $$T^Z_{RZ} = t_Z * K^Z_{RZ},$$
            $$T^Z_{RP} = t_Z * K^Z_{RP},$$
            где $t_Z$~--~норма времени на разработку ТЗ на программный продукт~(зависит от функционального назначения и степени новизны разрабатываемого программного продукта),~[чел.-дни].
            В нашем случае по таблице получаем значение~(группа новизны – В, функциональное назначение – технико-экономическое):
            $$t_Z = 37.$$
            $K^Z_{RZ}$~---~коэффициент, учитывающий удельный вес трудоемкости работ, выполняемых разработчиком постановки задачи на стадии ТЗ.
            В нашем случае~(совместная разработка с разработчиком ПО):
            $$K^Z_{RZ} = 0.65.$$
            $K^Z_{RP}$~---~коэффициент, учитывающий удельный вес трудоемкости работ, выполняемых разработчиком программного обеспечения на стадии ТЗ.
            В нашем случае~(совместная разработка с разработчиком постановки задач):
            $$K^Z_{RP} = 0.35.$$
            Тогда:
            $$\tau_{TZ} = 37 *~(0.35 + 0.65) = 37.$$

        \subsubsection{Трудоемкость разработки эскизного проекта}
            Расчёт трудоёмкости разработки эскизного проекта~($\tau_{EP}$)~[чел.-дни] производится по формуле:
            $$\tau_{EP} = T^E_{RZ} + T^E_{RP},$$
            где $T^E_{RZ}$~---~затраты времени разработчика постановки задачи на разработку эскизного проекта~(ЭП),~[чел.-дни];
            $T^E_{RP}$~---~затраты времени разработчика программного обеспечения на разработку ЭП,~[чел.-дни].
            Их значения рассчитываются по формулам:
            $$T^E_{RZ} = t_E * K^E_{RZ},$$
            $$T^E_{RP} = t_E * K^E_{RP},$$
            где $t_E$~--~норма времени на разработку ЭП на программный продукт~(зависит от функционального назначения и степени новизны разрабатываемого программного продукта),~[чел.-дни].
            В нашем случае по таблице получаем значение~(группа новизны – В, функциональное назначение – технико-экономическое):
            $$t_E = 77.$$
            $K^E_{RZ}$~---~коэффициент, учитывающий удельный вес трудоемкости работ, выполняемых разработчиком постановки задачи на стадии ЭП.
            В нашем случае~(совместная разработка с разработчиком ПО):
            $$K^E_{RZ} = 0.7.$$
            $K^E_{RP}$~---~коэффициент, учитывающий удельный вес трудоемкости работ, выполняемых разработчиком программного обеспечения на стадии ТЗ.
            В нашем случае~(совместная разработка с разработчиком постановки задач):
            $$K^E_{RP} = 0.3.$$
            Тогда:
            $$\tau_{EP} = 77 *~(0.3 + 0.7) = 77.$$

        \subsubsection{Трудоемкость разработки технического проекта}
            Трудоёмкость разработки технического проекта~($\tau_{TP}$)~[чел.-дни] зависит от функционального назначения программного продукта, количества разновидностей форм входной и выходной информации и определяется по формуле:
            $$\tau_{TP} = (t^T_{RZ} + t^T_{RP})*K_V*K_R,$$
            где $t^T_{RZ}$~---~норма времени, затрачиваемого на разработку технического проекта~(ТП) разработчиком постановки задач,~[чел.-дни];
            $t^T_{RP}$~---~норма времени, затрачиваемого на разработку ТП разработчиком ПО,~[чел.-дни].
            По таблице принимаем~(функциональное назначение~---~технико-экономическое планирование,
            количество разновидностей форм входной информации~---~1~(файл с текстом программы на исходном языке),
            количество разновидностей форм выходной информации~---~1~(файл с текстом программы на языке Scheme)):
            $$t^T_{RZ} = 30,$$
            $$t^T_{RP} = 8.$$
            $K_R$~---~коэффициент учета режима обработки информации. По таблице принимаем~(группа новизны~---~В, режим обработки информации~---~реальный масштаб времени):
            $$K_R = 1.26.$$
            $K_V$~---~коэффициент учета вида используемой информации, определяется по формуле:
            $$K_V = \dfrac {K_P*n_P + K_{NS}*n_{NS} + K_B*n_B} {n_P + n_{NS} + n_B },$$
            где $K_P$~---~коэффициент учета вида используемой информации для переменной информации;
            $K_{NS}$~---~коэффициент учета вида используемой информации для нормативно-справочной информации;
            $K_B$~---~коэффициент учета вида используемой информации для баз данных;
            $n_P$~---~количество наборов данных переменной информации;
            $n_{NS}$~---~количество наборов данных нормативно-справочной информации;
            $n_B$~---~количество баз данных.
            Коэффициенты находим по таблице~(группа новизны - В):
            $$K_P=1.00,$$
            $$K_{NS}=0.72,$$
            $$K_B=2.08.$$
            Количество наборов данных, используемых в рамках задачи:
            $$n_P=10,$$
            $$n_{NS}=0,$$
            $$n_B=0.$$
            Находим значение $K_V$:
            $$K_V = \dfrac{1.00*10+0.72*0+2.08*1}{10+0+1}=1.098.$$
            Тогда:
            $$\tau_{TP} = (30+8)*1.098*1.26 = 53.$$

        \subsubsection{Трудоемкость разработки рабочего проекта}
            Трудоёмкость разработки рабочего проекта~($\tau_{RP}$)~[чел.-дни] зависит от функционального назначения программного продукта, количества разновидностей форм входной и выходной информации, сложности алгоритма функционирования, сложности контроля информации, степени использования готовых программных модулей, уровня алгоритмического языка программирования и определяется по формуле:
            $$\tau_{RP} = (t^R_{RZ} + t^R_{RP})*K_K*K_R*K_Y*K_Z*K_{IA},$$
            где $t^R_{RZ}$~---~норма времени, затраченного на разработку рабочего проекта на алгоритмическом языке высокого уровня разработчиком постановки задач,~[чел.-дни].
            $t^R_{RP}$~---~норма времени, затраченного на разработку рабочего проекта на алгоритмическом языке высокого уровня разработчиком ПО,~[чел.-дни].
            По таблице принимаем~(функциональное назначение~---~технико-экономическое планирование,
            количество разновидностей форм входной информации~---~1~(файл с текстом программы на исходном языке),
            количество разновидностей форм выходной информации~---~1~(файл с текстом программы на языке Scheme)):
            $$t^R_{RZ} = 8,$$
            $$t^R_{RP} = 51.$$
            $K_K$~---~коэффициент учета сложности контроля информации.
            По таблице принимаем~(степень сложности контроля входной информации~---~12, степень сложности контроля выходной информации~---~22):
            $$K_K = 1.00.$$
            $K_R$~---~коэффициент учета режима обработки информации.
            По таблице принимаем~(группа новизны~---~В, режим обработки информации~---~реальный масштаб времени):
            $$K_R = 1.26.$$
            $K_Y$~---~коэффициент учета уровня используемого алгоритмического языка программирования. По таблице принимаем значение~(интерпретаторы, языковые описатели):
            $$K_Y = 0.8.$$
            $K_Z$~---~коэффициент учета степени использования готовых программных модулей. По таблице принимаем~(использование готовых программных модулей составляет менее 25%%):
            $$K_Z = 0.8.$$
            $K_{IA}$~---~коэффициент учета вида используемой информации и сложности алгоритма программного продукта, его значение определяется по формуле:
            $$K_IA = \dfrac {K'_P*n_P + K'_{NS}*n_{NS} + K'_B*n_B} {n_P + n_{NS} + n_B },$$
            где $K'_P$~---~коэффициент учета сложности алгоритма ПП и вида используемой информации для переменной информации;
            $K'_{NS}$~---~коэффициент учета сложности алгоритма ПП и вида используемой информации для нормативно-справочной информации;
            $K'_B$~---~коэффициент учета сложности алгоритма ПП и вида используемой информации для баз данных.
            $n_P$~---~количество наборов данных переменной информации;
            $n_{NS}$~---~количество наборов данных нормативно-справочной информации;
            $n_B$~---~количество баз данных.
            Коэффициенты находим по таблице~(группа новизны - В):
            $$K'_P=1.00,$$
            $$K'_{NS}=0.48,$$
            $$K'_B=0.4.$$
            Количество наборов данных, используемых в рамках задачи:
            $$n_P=10,$$
            $$n_{NS}=0,$$
            $$n_B=1.$$
            Находим значение $K_{IA}$:
            $$K_{IA} = \dfrac{1.00*10+0.48*0+0.4*1}{10+0+1}=0.945.$$
            Тогда:
            $$\tau_{RP} = (8+51)*1.00*1.26*0.8*0.8*0.945 = 45.$$

        \subsubsection{Трудоемкость выполнения стадии <<Внедрение>>}
            Расчёт трудоёмкости разработки технического проекта~($\tau_{V}$)~[чел.-дни] производится по формуле:
            $$\tau_{V} = (t^V_{RZ} + t^V_{RP})*K_K*K_R*K_Z,$$
            где $t^V_{RZ}$~---~норма времени, затрачиваемого разработчиком постановки задач на выполнение процедур внедрения программного продукта,~[чел.-дни];
            $t^V_{RP}$~---~норма времени, затрачиваемого разработчиком программного обеспечения на выполнение процедур внедрения программного продукта,~[чел.-дни].
            По таблице принимаем~(функциональное назначение~---~технико-экономическое планирование,
            количество разновидностей форм входной информации~---~1~(файл с текстом программы на исходном языке),
            количество разновидностей форм выходной информации~---~1~(файл с текстом программы на языке Scheme)):
            $$t^V_{RZ} = 9,$$
            $$t^V_{RP} = 11.$$
            Коэффициент $K_K$ и $K_Z$ были найдены выше:
            $$K_K=1.00,$$
            $$K_Z=0.8.$$
            $K_R$~---~коэффициент учета режима обработки информации. По таблице принимаем~(группа новизны~---~В, режим обработки информации~---~реальный масштаб времени):
            $$K_R = 1.26.$$
            Тогда:
            $$\tau_{V} = (9+11)*1.00*1.26*0.8= 21.$$

        Общая трудоёмкость разработки ПП:
        $$\tau_{PP} = 37+77+53+45+21= 233.$$

    \subsection{Расчет количества исполнителей}
    \label{subsec:slaves}
        Средняя численность исполнителей при реализации проекта разработки и внедрения ПО определяется соотношением:
        $$N=\dfrac {t} {F},$$
        где $t$~---~затраты труда на выполнение проекта (разработка и внедрение ПО); $F$~---~фонд рабочего времени.
        Разработка велась 5 месяцев с 1 января 2016 по 31 мая 2016.
        Количество рабочих дней по месяцам приведено в таблице~\ref{tabular:work_day}. Из таблицы получаем, что фонд рабочего времени $$F=96.$$
        \begin{table}[h!]
            \caption{Количество рабочих дней по месяцам\bigskip}
            \centering
            \label{tabular:work_day}
            \begin{tabular}{|c|c|c|}
                \hline
                \bf{Номер месяца} & \bf{Интервал дней}& \bf{Количество рабочих дней} \\ \hline
                1 & 01.01.2016~-~31.01.2016 & 15 \\ \hline
                3 & 01.02.2016~-~29.02.2016 & 20 \\ \hline
                4 & 01.03.2016~-~31.03.2016 & 21 \\ \hline
                5 & 01.04.2016~-~30.04.2016 & 21 \\ \hline
                6 & 01.05.2016~-~31.05.2016 & 19 \\ \hline
                \multicolumn{2}{|c|}{Итого} & 96 \\ \hline
            \end{tabular}
        \end{table}

        Получаем число исполнителей проекта:
        $$N=\dfrac{233}{96}=3$$

        Для реализации проекта потребуются 1 старший инженер и 2 простых инженера.

    \subsection{Ленточный график выполнения работ}
        На основе рассчитанных в главах \ref{subsec:trud}, \ref{subsec:slaves} трудоёмкости и фонда рабочего времени найдём количество рабочих дней, требуемых для выполнения каждого этапа разработка.
        Результаты приведены в таблице~\ref{tabular:tau_PP}.
        \begin{table}[ht!]
            %\small
            \caption{Трудоёмкость выполнения работы над проектом \bigskip}
            \centering

            \label{tabular:tau_PP}
            \begin{tabular}{|c|c|c|c|c|}
                \hline
                \bf{\specialcell{Номер \\ стадии}} & \bf{Название стадии} & \bf{\specialcell{Трудоёмкость\\$[$чел.-дни$]$}} & \bf{\specialcell{Удельный \\ вес $[$\%$]$}} & \bf{\specialcell{Количество\\ раб. дней}} \\ \hline
                1 &  Техническое задание    & 37  & 11  & 10 \\ \hline
                2 & Эскизный проект         & 77  & 24  & 23 \\ \hline
                3 & Технический проект      & 53  & 35  & 34 \\ \hline
                4 & Рабочий проект          & 45  & 25  & 24 \\ \hline
                5 & Внедрение               & 21  & 5   & 5  \\ \hline
                \multicolumn{2}{|c|}{Итого} & 233 & 100 & 96 \\ \hline
            \end{tabular}
        \end{table}

        Планирование и контроль хода выполнения разработки проводится по ленточному графику выполнения работ.
        По данным в таблице~\ref{tabular:tau_PP} в ленточный график (таблица ~\ref{tabular:lenta}), в ячейки столбца “продолжительности рабочих дней” заносятся времена, которые требуются на выполнение соответствующего этапа.
        Все исполнители работают одновременно.
        \begin{table}[ht!]
            \caption{Ленточный график выполнения работ \bigskip}
            \centering
            \label{tabular:lenta}
            \begin{tabular}{|c|c|c|c|c|c|c|c|c|c|c|c|c|c|c|c|c|c|c|c|c|c|c|c|c|}
                \hline

                & & \multicolumn{21}{|c|}{Календарные дни} \\ \cline{3-23}

                \parbox[t]{3mm}{\multirow{4}{*}[2em]{\rotatebox[origin=c]{90}{Номер стадии}}} &
                \parbox[t]{3.6mm}{\multirow{4}{*}[5.8em]{\rotatebox[origin=c]{90}{Продолжительность [раб.-дни]}}} &
                \rotatebox[origin=c]{90}{~11.01.2016~-~17.01.2016~} &
                \rotatebox[origin=c]{90}{~18.01.2016~-~24.01.2016~} &
                \rotatebox[origin=c]{90}{~25.01.2016~-~31.01.2016~} &
                \rotatebox[origin=c]{90}{~01.02.2016~-~07.02.2016~} &
                \rotatebox[origin=c]{90}{~08.02.2016~-~14.02.2016~} &
                \rotatebox[origin=c]{90}{~15.02.2016~-~21.02.2016~} &
                \rotatebox[origin=c]{90}{~22.02.2016~-~28.02.2016~} &
                \rotatebox[origin=c]{90}{~29.02.2016~-~06.03.2016~} &
                \rotatebox[origin=c]{90}{~07.03.2016~-~13.03.2016~} &
                \rotatebox[origin=c]{90}{~14.03.2016~-~20.03.2016~} &
                \rotatebox[origin=c]{90}{~21.03.2016~-~27.03.2016~} &
                \rotatebox[origin=c]{90}{~28.03.2016~-~03.04.2016~} &
                \rotatebox[origin=c]{90}{~04.04.2016~-~10.04.2016~} &
                \rotatebox[origin=c]{90}{~11.04.2016~-~17.04.2016~} &
                \rotatebox[origin=c]{90}{~18.04.2016~-~24.04.2016~} &
                \rotatebox[origin=c]{90}{~25.04.2016~-~01.05.2016~} &
                \rotatebox[origin=c]{90}{~02.05.2016~-~08.05.2016~} &
                \rotatebox[origin=c]{90}{~08.05.2016~-~15.05.2016~} &
                \rotatebox[origin=c]{90}{~16.05.2016~-~22.05.2016~} &
                \rotatebox[origin=c]{90}{~23.05.2016~-~29.05.2016~} &
                \rotatebox[origin=c]{90}{~30.05.2016~-~31.05.2016~}
                \\ \cline{3-23}

                & & \multicolumn{21}{|c|}{Количество рабочих дней} \\ \cline{3-23}

                  &    & 5 & 5 & 5 & 5 & 5 & 6 & 3 & 5 & 3 & 5 & 5 & 5 & 5 & 5 & 5 & 5 & 3 & 4 & 5 & 5 & 2 \\ \hline
                1 & 10 & 5 & 5 &   &   &   &   &   &   &   &   &   &   &   &   &   &   &   &   &   &   &   \\ \hline
                2 & 23 &   &   & 5 & 5 & 5 & 6 & 2 &   &   &   &   &   &   &   &   &   &   &   &   &   &   \\ \hline
                3 & 34 &   &   &   &   &   &   & 1 & 5 & 3 & 5 & 5 & 5 & 5 & 5 &   &   &   &   &   &   &   \\ \hline
                4 & 24 &   &   &   &   &   &   &   &   &   &   &   &   &   &   & 5 & 5 & 3 & 4 & 5 & 2 &   \\ \hline
                5 & 5  &   &   &   &   &   &   &   &   &   &   &   &   &   &   &   &   &   &   &   & 3 & 2 \\ \hline
            \end{tabular}
        %\end{sidewaystable}
        \end{table}

    \subsection{Определение себестоимости программной продукции}
        Затраты, образующие себестоимость продукции (работ, услуг), состоят из затрат на заработную плату исполнителям, затрат на закупку или аренду оборудования, затрат на организацию рабочих мест, и затрат на накладные расходы.

        В таблице~\ref{tabular:zarplata} приведены затраты на заработную плату и отчисления на социальное страхование в пенсионный фонд, фонд занятости и фонд обязательного медицинского страхования (30.5 \%).
        Для старшего инженера предполагается оклад в размере 120000 рублей в месяц, для инженера предполагается оклад в размере 100000  рублей в месяц.
        \begin{table}[ht!]
            %\small
            \caption{Затраты на зарплату и отчисления на социальное страхование \bigskip}
            \centering

            \label{tabular:zarplata}
            \begin{tabular}{|c|c|c|c|c|}
                \hline
                \bf{Должность} &
                \bf{\specialcell{Зарплата \\ в месяц}} &
                %\bf{\specialcell{Кол-во \\ работников}} &
                \bf{\specialcell{Рабочих \\ месяцев}} &
                \bf{\specialcell{Суммарная \\ зарплата}} &
                \bf{\specialcell{Затраты на \\ соц. нужды}} \\ \hline

                Старший инженер & 120000 & 5 & 600000 & 183000 \\ \hline
                Инженер & 100000 & 5 & 500000 & 152500 \\ \hline
                Инженер & 100000 & 5 & 500000 & 152500 \\ \hline
                \multicolumn{3}{|c|}{Суммарные затраты} & \multicolumn{2}{|c|}{2088000} \\ \hline
            \end{tabular}
        \end{table}

        Расходы на материалы, необходимые для разработки программной продукции, указаны в таблице~\ref{tabular:material}.

        \begin{table}[ht!]
            %\small
            \caption{Затраты на материалы \bigskip}
            \centering

            \label{tabular:material}
            \begin{tabular}{|c|c|c|c|c|}
                \hline
                \bf{\specialcell{Наименование \\ материала}} &
                \bf{\specialcell{Единица \\ измерения}} &
                %\bf{\specialcell{Кол-во \\ работников}} &
                \bf{\specialcell{Кол-во}} &
                \bf{\specialcell{Цена за \\ единицу, руб.}} &
                \bf{\specialcell{Сумма, руб.}} \\ \hline

                Бумага А4 & Пачка 400 л. & 2 & 200 & 400 \\ \hline
                \specialcell{Картридж для \\ принтера HP F4275} & Шт. & 2 & 675 & 1350 \\ \hline
                \multicolumn{4}{|c|}{Суммарные затраты} & \multicolumn{1}{|c|}{1750} \\ \hline
            \end{tabular}
        \end{table}

        В работе над проектом используется специальное оборудование~---~персональные электронно-вычислительные машины (ПЭВМ) в количестве 9 шт.
        Стоимость одной ПЭВМ составляет 90000 рублей.
        Месячная норма амортизации K = 2,7\%.
        Тогда за 5 месяцев работы расходы на амортизацию составят $P = 90000 * 3 *  0.027 * 5 = 36450$~рублей.

        Накладные расходы рассчитываются по следующе формуле:
        $$C_n=A_n * C_z$$
        $$N=2.1*1600000=3360000$$

        Общие затраты на разработку программного продукта (ПП) составят

        {\centering{$2088000+1750+36450+3360000=5486200$ рублей.}

        }


    \subsection{Определение стоимости программной продукции}
        Для определения стоимости работ необходимо на основании плановых сроков выполнения работ и численности исполнителей рассчитать общую сумму затрат на разработку программного продукта.
        Если ПП рассматривается и создается как продукция производственно-технического назначения,
        допускающая многократное тиражирование и отчуждение от непосредственных разработчиков, то ее цена~$P$ определяется по формуле:
        $$P = K*C+Pr,$$
        где $C$~---~затраты на разработку ПП (сметная себестоимость);
        $K$~---~коэффициент учёта затрат на изготовление опытного образца ПП как продукции производственно-технического назначения~($K=1.1$);
        $Pr$~---~нормативная прибыль, рассчитываемая по формуле:
        $$Pr= \frac {C * \rho_N} {100},$$
        где $\rho_N$~---~норматив рентабельности, $\rho_N=30\%$;

        Получаем стоимость программного продукта:

        {\centering$P=1.1*5486200 + 5486200*0.3=7680680$~рублей.

        }
    \subsection{Расчет экономической эффективности}
        Основными показателями экономической эффективности является чистый дисконтированный доход~(NPV) и срок окупаемости вложенных средств.
        Чистый дисконтированный доход определяется по формуле:
        $$NPV=\sum_{t=0}^T (R_t-Z_t) * \dfrac{1}{(1+E)^t},$$
        где $T$~---~горизонт расчета по месяцам;
        $t$~---~период расчета;
        $R_t$~---~результат, достигнутый на $t$ шаге (стоимость);
        $Z_t$~---~текущие затраты (на шаге $t$);
        $E$~---~приемлемая для инвестора норма прибыли на вложенный капитал.

        На момент начала 2016 года, ставка рефинансирования 11\% годовых~(ЦБ РФ), что эквивалентно (0.916\% в месяц). В виду особенности разрабатываемого продукта он может быть продан лишь однократно.
        Отсюда получаем $$E=0.00916.$$

        В таблице~\ref{tabular:pdd} находится расчёт чистого дисконтированного дохода. График его изменения приведён на рисунке~\ref{pic:pdd}.

        \begin{table}[ht!]
            %\small
            \caption{Расчёт чистого дисконтированного дохода \bigskip}
            \centering

            \label{tabular:pdd}
            \begin{tabular}{|c|c|c|c|c|}
                \hline
                \bf{\specialcell{Месяц}} &
                \bf{\specialcell{Текущие \\затраты, руб.}} &
                %\bf{\specialcell{Кол-во \\ работников}} &
                \bf{\specialcell{Затраты с начала \\ года, руб.}} &
                \bf{\specialcell{Текущий \\ доход, руб.}} &
                \bf{\specialcell{ЧДД, руб.}} \\ \hline

                Январь  & 1096890 & 1096890 & 0       & -1096890 \\ \hline
                Февраль & 1096890 & 2193780 & 0       & -2183823.7 \\ \hline
                Март    & 1096890 & 3290670 & 0       & -3260891.4 \\ \hline
                Апрель  & 1096890 & 4387560 & 0       & -4328182.7 \\ \hline
                Мая     & 1098640 & 5486200 & 7680680 & 2019802 \\ \hline

            \end{tabular}
        \end{table}

        \begin{figure}[h!]
            \centering
            \begin{tikzpicture}[scale=1]
                \begin{axis}[ylabel=ЧДД (руб.), xlabel=Количество месяцев с начала проекта,
                ] %\tiny
                    \addplot coordinates {
                        (1, -1096890)
                        (2, -2183823.7)
                        (3, -3260891.4)
                        (4, -4328182.7)
                        (5, 2019802)
                    };
                \end{axis}
            \end{tikzpicture}
            \caption{График изменения чистого дисконтированного дохода}
            \label{pic:pdd}
        \end{figure}

        Согласно проведенным расчетам, проект является рентабельным.
        Разрабатываемый проект позволит превысить показатели качества существующих систем и сможет их заменить.
        Итоговый ЧДД составил: $2019802$ рубля.

    \subsection{Результаты}
        В рамках организационно-экономической части был спланирован календарный график проведения работ по созданию подсистемы поддержки проведения диагностики промышленных, а также были проведены расчеты по трудозатратам.
        Были исследованы и рассчитаны следующие статьи затрат: материальные затраты; заработная плата исполнителей; отчисления на социальное страхование; накладные расходы.

        В результате расчетов было получено общее время выполнения проекта, которое составило $96$ рабочих дней, получены данные по суммарным затратам на создание и разработку функционального языка, которые составили $5486200$ рублей.
        Согласно проведенным расчетам, проект является рентабельным.
        Цена данного программного проекта составила $7680680$ рублей, итоговый ЧДД составил $2019802$ рублей.

    \clearpage
    \section*{\hfillПРИЛОЖЕНИЕ\hfill}
\addcontentsline{toc}{section}{Приложение}
    \subsection*{ПРИЛОЖЕНИЕ А. Функция вычисления дня недели}
    \addcontentsline{toc}{subsection}{Приложение А. Функция вычисления дня недели}
    \lstinputlisting{../examples/1.sm}
    В этом программе определена функция вычисления дня недели по дате.

    В этом примере используются определения внутри функции.
    Так, три переменные \verb,a,, \verb,y, и \verb,m, определены и доступны только внутри функции \verb,day-of-week,.

    В процессе выполнения такой программы будут возвращены четыре значения: 2 (вторник), 0 (воскресенье), 2 (вторник), 3 (среда).

    \subsection*{ПРИЛОЖЕНИЕ Б. Функция замены всех нулей на единицы}
    \addcontentsline{toc}{subsection}{Приложение Б. Функция замены всех нулей на единицы}
    \lstinputlisting{../examples/2.sm}

    В этом примере используется сопоставление с образцом.
    Если аргумент функции --- пустой список, то она возвращает пустой список.
    Если первым элементом списка является число 0, то возвращается список, первым элементом которого является 1, а остаток этого списка вычисляется рекурсивно.
    Если шаблоны, описанные в первых двух строках не подошли, будет использован шаблон в третьей строке.
    Он более универсален.
    Результатом вычисления в этом случае будет список, состоящий из значения аргумента \verb,x, и списка, вычисленного рекурсивно.

    Вычисленные в ходе выполнения программы списки выглядят следующим образом:
    \\\verb,[1 2 7 1 5],
    \\\verb,[1 1 1 1 1],

    \subsection*{ПРИЛОЖЕНИЕ В. Функция подсчёта количества вхождений}
    \addcontentsline{toc}{subsection}{Приложение В. Функция подсчёта количества вхождений}
    \lstinputlisting{../examples/3.sm}

    Функция \verb,count, возвращает количество вхождений первого аргумента, в список, являющийся вторым аргументом.
    
    В первой строке мы видим определение функции \verb,count, от \verb,x, и пустого списка.
    В этом случае количество вхождений \verb,x, в список равно нулю.
    Во второй строке мы видим определение функции \verb,count, от \verb,x, и списка, первым элементом которого является \verb,x,.
    В данном случае нам нужно прибавить единицу к количеству вхождений \verb,x, в <<хвост>> списка \verb,xs,.
    И, наконец, в третьей строке определена функция \verb,count, от \verb,x, и списка, первым элементом которого является \verb,y,.
    Так как в строке выше мы предусмотрели равенство \verb,x, и первого элемента списка, то в этой строке определения мы можем быть уверены, что \verb,x, не равен \verb,y,.

    Результат выполнения программы:
    \\\verb,2,
    \\\verb,0,
    \\\verb,3,
    \subsection*{ПРИЛОЖЕНИЕ Г. Функция вычисления факториала числа}
    \addcontentsline{toc}{subsection}{Приложение Г. Функция вычисления факториала числа}
    \lstinputlisting{../examples/4.sm}

    \subsection*{ПРИЛОЖЕНИЕ Д. Функция вычисления суммы произвольного количества аргументов}
    \addcontentsline{toc}{subsection}{Приложение Д. Функция вычисления суммы произвольного количества аргументов}
    \lstinputlisting{../examples/5.sm}

    В данном примере определена функция вычисления факториала числа.
    Запись этой функции похожа на математическую запись этой функции.
    В случае равенства первого аргумента нулю --- будет возвращена единица,
    иначе будет посчитано произведение числа \verb,n, на факториал от числа \verb,n - 1,.

    В четвёртой строке программы указано, что эту функцию необходимо мемоизировать.

    Значение, вычисленное для числа 5 --- 120.
    Значение, вычисленное для числа 10 --- 3628800.

    Стоит отметить, что при вычислении факториала 10, высчитываются только значения этой функции для чисел 9, 8, 7, 6.
    Значение для числа 5 уже вычисленно при первом вызове функции.

    \subsection*{ПРИЛОЖЕНИЕ Е. Функция замены элементов списка}
    \addcontentsline{toc}{subsection}{Приложение Е. Функция замены элементов списка}
    \lstinputlisting{../examples/6.sm}

    В этой программе описана функция \verb,replace,, принимающая три аргумента:
    \begin{itemize}
        \item предикат (\verb,pred?,) --- функцию, проверяющую какое-либо условие;
        \item функцию одного аргумента (\verb,proc,), с помощью которой будет осуществлено преобразование;
        \item список.
    \end{itemize}

    В случае, если список пуст, возвращается пустой список.

    Для списка, не являющегося пустым будет осуществлена проверка первого элемента с помощью предиката.
    В случае, если условие, описанное в предикате выполнено, будет возвращён список, первым элементом которого будет являться изменённый функцией \verb,proc, первый элемент списка (\verb,x,).
    <<Хвост>> итогового списка вычисляется рекурсивно для списка \verb,xs,.

    С шестой по восьмую строках программы записан вызов функции \verb,replace, от стандартного предиката \verb,zero?, (проверяющего, является ли входной аргумент нулём), $\lambda$-функции, увеличивающей число на единицу и списка из 5 элементов.
    Результатом этого вызова будет список \verb,[1 1 2 3 0],.

    С десятой до двенадцатую строки программы описывается вызов функции \verb,replace, от стандартной функции \verb,odd?, (осущестляющей проверку на нечётность), $\lambda$-функции, удваивающей свой аргумент и списка из шести чисел.
    Результатом этого вызова будет список \verb,[1 1 2 3 0],.

    С четырнадцатой по шестнадцатую строку программы дано описание вызова функции от $\lambda$-функции, проверяющей, является ли число большим нуля, функции вычисления экспоненты числа и списка из семи аргументов.
    Результат этого вызова: \verb,[0 1 0.3678794 2 0.1353352 3 0.0497870],.

    \subsection*{ПРИЛОЖЕНИЕ Ё. Функция повтора элемента}
    \addcontentsline{toc}{subsection}{Приложение Ё. Функция повтора элемента}
    \lstinputlisting{../examples/7.sm}

    Результат выполнения программы:
    \\ \verb,["a" "a" "a" "a" "a"],
    \\ \verb,[["a" "b"] ["a" "b"] ["a" "b"]],
    \\ \verb,[],

    \subsection*{ПРИЛОЖЕНИЕ Ж. Функция повторения списка}
    \addcontentsline{toc}{subsection}{Приложение Ж. Функция повторения списка}
    \lstinputlisting{../examples/8.sm}

    Результат выполнения программы:
    \\ \verb,[0 1 0 1 0 1],
    \\ \verb,["a" "b" "c" "a" "b" "c" "a" "b" "c" "a" "b" "c" "a" "b" "c"],
    \\ \verb,[],

    \subsection*{ПРИЛОЖЕНИЕ З. Функция AND для произвольного количества аргументов}
    \addcontentsline{toc}{subsection}{Приложение З. Функция AND для произвольного количества аргументов}
    \lstinputlisting{../examples/9.sm}

    Результат выполнения программы:
    \\ \verb,#f,
    \\ \verb,#f,
    \\ \verb,#f,
    \\ \verb,#t,
    \\ \verb,#t,

    \subsection*{ПРИЛОЖЕНИЕ И. Вставка кода на языке Scheme}
    \addcontentsline{toc}{subsection}{Приложение И. Вставка кода на языке Scheme}
    Пример вставки кода на языке Scheme.
    Вставляемые функции --- функции сортировки выбором и вставками.
    \lstinputlisting{../examples/10.sm}

    Результат выполнения программы:
    \\ \verb,[0 1 2 3 4 5 6 7 8 9],
    \\ \verb,[0 1 2 3 4 5 6 7 8 9],

    \subsection*{ПРИЛОЖЕНИЕ К. Функция проверки числа на простоту}
    \addcontentsline{toc}{subsection}{Приложение К. Функция проверки числа на простоту}
    \lstinputlisting{../examples/11.sm}

    Результат выполнения программы:
    \\ \verb,#t,
    \\ \verb,#f,
    \\ \verb,#t,

    \subsection*{ПРИЛОЖЕНИЕ Л. Функции НОД и НОК}
    \addcontentsline{toc}{subsection}{Приложение Л. Функции НОД и НОК}
    \lstinputlisting{../examples/12.sm}

    Результат выполнения программы:
    \\ \verb,154,
    \\ \verb,12,

    \subsection*{ПРИЛОЖЕНИЕ М. Метод половинного сечения}
    \addcontentsline{toc}{subsection}{Приложение М. Метод половинного сечения}
    \lstinputlisting{../examples/13.sm}

    Результат выполнения программы:
    \\ \verb,-1.5703125,

    \subsection*{ПРИЛОЖЕНИЕ Н. Функция вычисляющая диапазон значений}
    \addcontentsline{toc}{subsection}{Приложение Н. Функция вычисляющая диапазон значений}
    \lstinputlisting{../examples/14a.sm}

    Результат выполнения программы:
    \\ \verb,[0 3 6 9],

    \subsection*{ПРИЛОЖЕНИЕ О. Функция разворачивания списков}
    \addcontentsline{toc}{subsection}{Приложение О. Функция разворачивания списков}
    \lstinputlisting{../examples/15.sm}

    Результат выполнения программы:
    \\ \verb,[1 2 3 4 5 6 7 8 9],

    \subsection*{ПРИЛОЖЕНИЕ П. Функция проверки вхождения в список}
    \addcontentsline{toc}{subsection}{Приложение П. Функция проверки вхождения в список}
    \lstinputlisting{../examples/16a.sm}

    Результат выполнения программы:
    \\ \verb,#t,
    \\ \verb,#f,

    
    \clearpage
    \begin{thebibliography}{0}
    \bibitem{TIOBE}\verb$http://www.tiobe.com/tiobe_index$
    \bibitem{p_c_lisp}Peter Seibel. Practical Common Lisp. Publication Date: April 6, 2005
    \bibitem{scheme_doc}\verb$http://www.schemers.org/Documents/Standards/R5RS/HTML/$
    \bibitem{scheme_pl}R. Kent Dybvig. The Scheme Programming Language, Fourth Edition. The MIT Press. 2009
    \bibitem{prolog}Information technology — Programming languages — Prolog. 1995
    \bibitem{unify}http://www.nsl.com/misc/papers/martelli-montanari.pdf
    \bibitem{prolog_bratko}Братко, Иван. Алгоритмы искусственного интеллекта на языке PROLOG, 3-е издание. Пер. с англ. --- М. : Издательский дом <<Вильяме>>, 2004.
    \bibitem{python}A Byte of Python, http://python.swaroopch.com/
    \bibitem{ocaml}Yaron Minsky, Anil Madhavapeddy, Jason Hickey. Real world OCaml. 2013
    \bibitem{haskell}Miran Lipovača. LEARN YOU A HASKELL FOR GREAT GOOD!. 2011 

    \bibitem{evm} Орлов С.А., Цилькер Б.Я. Организация ЭВМ и систем: Учебник для вузов. 2-е изд. --- СПб.: Питер, 2011
    \bibitem{langs} Сергей Александрович Орлов. Теория и практика языков программирования: Учебник для вузов. Стандарт 3-го поколения. --- СПб.: Питер, 2014
    \bibitem{skor} Скоробогатов С.Ю. Лекции по курсу <<Компиляторы>>, 2014.
    \bibitem{economic_sajin} Арсеньев В.В., Сажин Ю.Б. Методические указания к выполнению организационно-экономической части дипломных проектов по созданию программной продукции. М.: изд. МГТУ им. Баумана, 1994. 52 с. 2.
    \bibitem{economic_smirnov} Под ред. Смирнова С.В. Организационно-экономическая часть дипломных проектов исследовательского профиля. М.: изд. МГТУ им. Баумана, 1995. 100 с.
\end{thebibliography}


\clearpage
\end{document}