\documentclass[12pt]{article}

\usepackage{ucs}
\usepackage{amsmath}                    % gather* для вывода формул по центру страницы
\usepackage[utf8x]{inputenc}        % Включаем поддержку UTF8
\usepackage[russian]{babel}         % Включаем пакет для поддержки русского языка

\usepackage[
    left=2cm,           % Поле левое : 200 мм
    right=2cm,          % Поле правое : 200 мм
    top=2cm,            % Поле верхнее: 200 мм
    bottom=2cm,         % Поле нижнее : 200 мм
    bindingoffset=0cm]{geometry}

\setlength{\parindent}{1.25cm}          % Абзацный отступ: 1,25 см
\usepackage{indentfirst}                % 1-й абзац имеет отступ

\usepackage[pdftex]{graphicx, color}
\usepackage{color}
\usepackage{tikz}
\usepackage{url}            % использование URL в библиографии
\usepackage{listings}           % использование листингов кода
\usepackage[nooneline]{caption}
\captionsetup[table]{justification=raggedleft}
\captionsetup[figure]{justification=centering,labelsep=endash}
%\usepackage{array}

\usepackage[nodisplayskipstretch]{setspace}
\setstretch{1.5}

\usepackage{caption}
\usepackage{graphicx}
\usepackage{subcaption}
\usepackage{cases}
\usepackage{placeins}
\usepackage{longtable} 
\usepackage{multirow}
\usepackage{hhline}

\renewcommand{\baselinestretch}{1.5}

\definecolor{grey}{HTML}{EEEEEE}
% вставка листингов с кодом
\lstset{inputencoding=utf8x,
        extendedchars=false,
        keepspaces=true,
        basicstyle=\small\sffamily,
        numbers=left,
        numberstyle=\tiny,
        numbersep=5pt,
        stepnumber=1,                   % размер шага между двумя номерами строк
        backgroundcolor=\color{white},
        showspaces=false,            % показывать или нет пробелы специальными отступами
        showstringspaces=false,      % показывать или нет пробелы в строках
        showtabs=false,             % показывать или нет табуляцию в строках
        frame=none,
        tabsize=2,                 % размер табуляции по умолчанию равен 2 пробелам
        captionpos=t,              % позиция заголовка вверху [t] или внизу [b] 
        breaklines=true,           % автоматически переносить строки (да\нет)
        breakatwhitespace=false} % переносить строки только если есть пробел
        
%\input{listings-python.prf}
\renewcommand{\lstlistingname}{Листинг}

\setcounter{tocdepth}{4}    % chapter, section, subsection, subsubsection и paragraph
\setcounter{secnumdepth}{4}

\parindent=1,25cm               % красная строка = 1 см
\usepackage{enumitem}
\setlist[enumerate,1]{leftmargin=2.25cm}
\setlist[itemize]{leftmargin=2.25cm}
\graphicspath{{pics/}}
\DeclareGraphicsExtensions{{.jpg}}
\usepackage{color, colortbl}

\begin{document}
    %\pgfplotsset{compat=1.8}

\thispagestyle{empty}
\newpage
{
\centering


\textbf{
МОСКОВСКИЙ ГОСУДАРСТВЕННЫЙ ТЕХНИЧЕСКИЙ УНИВЕРСИТЕТ ИМЕНИ Н. Э. БАУМАНА \\
Факультет информатики и систем управления \\
Кафедра теоретической информатики и компьютерных технологий}

\vfill

\hfill\parbox{7cm} {
УТВЕРЖДАЮ:\\
Заведующий кафедрой ИУ-9 \hfill \\
\underline{\hspace{4cm}}(И.П. Иванов)\hfill \\
<<\underline{\hspace{0.5cm}}>>\underline{\hspace{3cm}}201\underline{\hspace{0.5cm}}г.\hfill \\
}

\bigskip
\bigskip
\bigskip
\bigskip
\bigskip
\bigskip
\bigskip
\bigskip

\vfill

{\large\bf Обзор} \\
к дипломному проекту \\
<<Функциональный язык программирования с динамической типизацией и ML-подобным синтаксисом>>

\vfill

\hfill\parbox{7cm} {
Исполнитель: Ю.А. Волкова \\
Группа: ИУ9-111
}

\bigskip
\bigskip
\bigskip
\bigskip
\bigskip
\bigskip
\bigskip

\vfill

Руководитель \\
квалификационной работы \\
А.В. Дубанов

\vspace{\fill}
}

\clearpage

    \tableofcontents
    \clearpage

    % Введение, актуальность -- 2
\section*{Введение}
    Анализ текущего состояния в разработке языков программирования показал, что существует необходимость в языке программирования, ориентированном на быстрой разработке сценариев (скриптов), первоначальном обучении программированию и исследовательском программировании.

    Для этих целей, на наш взгляд, должен существовать функциональный язык с <<дружелюбным>> синтаксисом, который подразумевает инфиксную нотацию в записи арифметических выражений.

    В настоящее время существует много скриптовых языков программирования, однако функциональных среди них мало.
    Функциональных языков тоже много.
    Но скриптовых среди них мало.
% Языки-прототипы -- 3
\section{Языки-прототипы}
\addcontentsline{toc}{section}{Языки-прототипы}
% Синтаксис ФЯП -- 3
\addcontentsline{toc}{section}{Синтаксис функционального языка программирования}
    Для описания синтаксиса языка используются расширенная форма Бэкуса-Наура (РБНФ). 
    Альтернатива обозначается символом '|'. 
    '*' после выражения означает, что оно может быть включено $0$ и более раз, '+' - $1$ и более, '?' - $0$ или $1$ раз%\cite{skor}.
    Нетерминальные символы начинаются с заглавной буквы. 
    Терминальные либо начинаются малой буквой, либо состоят целиком из заглавных букв.

    Язык является регистрозависимым.

    \subsection{Словарь и представление}
        Существуют следующие виды токенов: идентификатор, число, символ, строка, операторы и ключевые слова.
        Пробельные символы игнорируются, если они не существенны для разделения двух последовательных токенов.

        \begin{itemize}
            \item Идентификаторы -- последовательности символов, исключающие пробельные символы, точки, запятые, скобки, кавычки, вертикальную черту.
            Второй вариант записи идентификаторов -- внутри двух вертикальных черт -- подразумевает возможность использовать любые символы.

            \lstinputlisting{../syntax/ident}
            \item Число -- целочисленная или вещественная константа. Типом числа считается минимальный тип, содержащий значение этого числа. 
            Если константа начинается с $'0x'$, то число рассматривается, как записанное в шестнадцатеричной системе счисления. 
            Иначе - в десятичной.
            
            \lstinputlisting{../syntax/number}
            \item Строки -- последовательности символов, заключённые в одинарные (') или двойные(") кавычки.
                
            %\lstinputlisting{lst_str}
            \item Операторы и ключевые слова -- специальные символы, пары символов и слова, зарезервированные системой.

            Список таких символов: $ + - < > * / = ~ , " ' . : | [ ] ( ) { } \ -- -> <-$ if scheme map filter reduce eval
        \end{itemize}

% Способы реализации ЯП (что-то там с DSL Clojure?) -- 5
\section{Способы реализации языка программирования}
\addcontentsline{toc}{section}{Способы реализации языка программирования}
% Компоненты среды разработки -- 6
%\section{Компоненты среды разработки}
% Примеры кода
\section{Примеры кода}
\addcontentsline{toc}{section}{Примеры кода}
    \lstinputlisting{../examples/sample1.sm}
% 2 + 3 + 3 + 5 + 6 = 19
    \begin{thebibliography}{0}
    \bibitem{TIOBE}\verb$http://www.tiobe.com/tiobe_index$
    \bibitem{p_c_lisp}Peter Seibel. Practical Common Lisp. Publication Date: April 6, 2005
    \bibitem{scheme_doc}\verb$http://www.schemers.org/Documents/Standards/R5RS/HTML/$
    \bibitem{scheme_pl}R. Kent Dybvig. The Scheme Programming Language, Fourth Edition. The MIT Press. 2009
    \bibitem{prolog}Information technology — Programming languages — Prolog. 1995
    \bibitem{unify}http://www.nsl.com/misc/papers/martelli-montanari.pdf
    \bibitem{prolog_bratko}Братко, Иван. Алгоритмы искусственного интеллекта на языке PROLOG, 3-е издание. Пер. с англ. --- М. : Издательский дом <<Вильяме>>, 2004.
    \bibitem{python}A Byte of Python, http://python.swaroopch.com/
    \bibitem{ocaml}Yaron Minsky, Anil Madhavapeddy, Jason Hickey. Real world OCaml. 2013
    \bibitem{haskell}Miran Lipovača. LEARN YOU A HASKELL FOR GREAT GOOD!. 2011 

    \bibitem{evm} Орлов С.А., Цилькер Б.Я. Организация ЭВМ и систем: Учебник для вузов. 2-е изд. --- СПб.: Питер, 2011
    \bibitem{langs} Сергей Александрович Орлов. Теория и практика языков программирования: Учебник для вузов. Стандарт 3-го поколения. --- СПб.: Питер, 2014
    \bibitem{skor} Скоробогатов С.Ю. Лекции по курсу <<Компиляторы>>, 2014.
    \bibitem{economic_sajin} Арсеньев В.В., Сажин Ю.Б. Методические указания к выполнению организационно-экономической части дипломных проектов по созданию программной продукции. М.: изд. МГТУ им. Баумана, 1994. 52 с. 2.
    \bibitem{economic_smirnov} Под ред. Смирнова С.В. Организационно-экономическая часть дипломных проектов исследовательского профиля. М.: изд. МГТУ им. Баумана, 1995. 100 с.
\end{thebibliography}


\clearpage
\end{document}