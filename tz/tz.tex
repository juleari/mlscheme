% utf-8 ru
\documentclass[12pt,a4paper,oneside]{extarticle}
    \righthyphenmin=2 %минимально переносится 2 символа %%%
    \sloppy

% Рукопись оформлена в соответствии с правилами оформления 
% электронной версии авторского оригинала, 
% принятыми в Издательстве МГТУ им. Н.Э. Баумана.

\usepackage{geometry} % А4, примерно 28-31 строк(а) на странице 
    \geometry{paper=a4paper}
    \geometry{includehead=false} % Нет верх. колонтитула
    \geometry{includefoot=true}  % Есть номер страницы
    \geometry{bindingoffset=0mm} % Переплет    : 0  мм
    \geometry{top=20mm}          % Поле верхнее: 20 мм
    \geometry{bottom=25mm}       % Поле нижнее : 25 мм 
    \geometry{left=25mm}         % Поле левое  : 25 мм
    \geometry{right=25mm}        % Поле правое : 25 мм
    \geometry{headsep=10mm}  % От края до верх. колонтитула: 10 мм
    \geometry{footskip=20mm} % От края до нижн. колонтитула: 20 мм 

\usepackage{changepage}

\usepackage{cmap}
\usepackage[T2A]{fontenc}
\usepackage[utf8x]{inputenc}
\usepackage[english,russian]{babel}
\usepackage{misccorr}

\usepackage{amsmath}
\usepackage{amsfonts}
\usepackage{amssymb}

%\usepackage{cm-super} %человеческий рендер русских шрифтов

\setlength{\parindent}{1.25cm}  % Абзацный отступ: 1,25 см
\usepackage{indentfirst}        % 1-й абзац имеет отступ

\usepackage{setspace} 
\usepackage{multirow} 
\usepackage{hhline}

\onehalfspacing % Полуторный интервал между строками


\usepackage{pgfplots}

\usepackage{listings}
\lstset{basicstyle=\footnotesize}

\begin{document}
\pgfplotsset{compat=1.8}

\thispagestyle{empty}
\newpage
{
\centering


\textbf{
МОСКОВСКИЙ ГОСУДАРСТВЕННЫЙ ТЕХНИЧЕСКИЙ УНИВЕРСИТЕТ ИМЕНИ Н. Э. БАУМАНА \\
Факультет информатики и систем управления \\
Кафедра теоретической информатики и компьютерных технологий}

\vfill

\hfill\parbox{7cm} {
УТВЕРЖДАЮ:\\
Заведующий кафедрой ИУ-9 \hfill \\
\underline{\hspace{4cm}}(И.П. Иванов)\hfill \\
<<\underline{\hspace{0.5cm}}>>\underline{\hspace{3cm}}201\underline{\hspace{0.5cm}}г.\hfill \\
}

\bigskip
\bigskip
\bigskip
\bigskip
\bigskip
\bigskip
\bigskip
\bigskip

\vfill

{\large\bf Техническое задание} \\
на дипломный проект \\
<<Функциональный язык программирования с динамической типизацией и ML-подобным синтаксисом>>

\vfill

\hfill\parbox{7cm} {
Исполнитель: Ю.А. Волкова \\
Группа: ИУ9-111
}

\bigskip
\bigskip
\bigskip
\bigskip
\bigskip
\bigskip
\bigskip

\vfill

\hfill\parbox{7cm} {
СОГЛАСОВАНО
}

\vfill

Руководитель \\
квалификационной работы \\
\vfill

\hfill\parbox{7cm} {
\underline{\hspace{4cm}}(А.В. Дубанов)\hfill \medskip\\
<<\underline{\hspace{0.5cm}}>>\underline{\hspace{3cm}}201\underline{\hspace{0.5cm}}г.\hfill \\
}

\vspace{\fill}
}

\clearpage

\section{Введение.}
    1.1. Наименование проекта <<Функциональный язык программирования с динамической типизацией и ML-подобным синтаксисом>>
    
    1.2. Исполнитель дипломного проекта --- Ю.А. Волкова.

\section{Основания для разработки.}
    2.1. Основанием для разработки дипломного проекта является решение кафедры ИУ-9 от <<\underline{\hspace{0.5cm}}>>\underline{\hspace{3cm}}200\underline{\hspace{0.5cm}}г., протокол №\underline{\hspace{0.75cm}}.

\section{Используемые понятия.}
    \begin{tabular}{| p{3cm} | p{11cm} |}\hline
    Понятие         & Определение \\ \hline
    ЯП              & Язык Программирования - формальная знаковая система, предназначенная для записи компьютерных программ \\ \hline
    Система типов   & Совокупность правил в языках программирования, назначающих свойства различным конструкциям, формирующим программу. \\ \hline
    Статическая типизация  & Приём в программировании, при котором переменная связывается с типом при создании.\\ \hline
    Динамическая типизация & Приём в программировании, при котором переменная связывается с типом в момент присваивания значения. \\ \hline
    Сценарий (скрипт)      &  Программа, имеющая дело с готовыми программными компонентами. \\ \hline
    Интерактивная отладка  & Режим отладки программы, позволяющий выполнить её, останавливаясь в точках останова, расставленных в коде программы.\\ \hline
    Метапрограмми-рование  & Вид программирования, связанный с созданием программ, которые порождают другие программы как результат своей работы.\\ \hline
    Гомоиконность          & Свойство языка программирования, подразумевающее единство кода и данных\\ \hline
    Интерпретатор для языка L & Программа, осуществляющая выполнение программ, написанных на языке L.\\ \hline
    Компилятор из языка S в язык T & Программа, осуществляющая перевод программ, с языка S на язык T.\\ \hline
    Целевой ЯП              & Результирующий язык, на который осуществляется перевод программы. \\ \hline
    \end{tabular}

\section{Цель и назначения разработки.}
    \subsection{Назначение}
        Функциональный ЯП общего назначения, в большей степени ориентированный на:
            \begin{itemize}
                \item быструю разработку сценариев (скриптов),
                \item первоначальное обучение программированию,
                \item исследовательское программирование.
            \end{itemize}
    \subsection{Цель}
        Разработка языка программирования и его реализация на основе существующего компилятора или интерпретатора какого-либо функционального ЯП.

\section{Требования к программному комплексу.}
    \subsection{Требования к языку:}
        \begin{itemize}
            \item функциональная парадигма -- основная,
            \item возможность написания программ в рамках другой парадигмы,
            \item запись выражений в традиционной инфиксной нотации с возможностью определения новых операторов и их приоритета,
            \item отступы не должны быть существенны,
            \item минимум разделителей и ключевых слов,
            \item функции с переменным числом аргументов,
            \item динамическая типизация,
            \item возможность интерактивной отладки,
            \item развитые средства метапрограммирования,
                \begin{itemize}
                    \item гомоиконность,
                    \item макросы
                \end{itemize}
            \item возможность подключения внешних библиотек,
            \item возможность включения в программу кода на целевом ЯП,
            \item поддержка многопоточных вычислений.
        \end{itemize}
    \subsection{Самоприменимость.}
    \subsection{Составные части:}
        \begin{itemize}
            \item лексический анализатор,
            \item синтаксический анализатор,
            \item семантический анализатор,
            \item генератор кода на ЯП, на основании реализации которого создаётся данный язык,
            \item стандартная библиотека,
            \item документация,
            \item примеры и тесты.
        \end{itemize}
    \subsection{Надёжность.}
        \begin{itemize}
            \item Модули программы должны корректно выполнять преобразования исходного кода на созданном языке в эквивалентную ей программу на целевом ЯП.
            \item Работоспособность компонентов программного комплекса должна быть проверена на тестовых наборах программ.
        \end{itemize}
    \subsection{Условия эксплуатации.}
        \begin{itemize}
            \item Условия эксплуатации должны соответствовать требованиям по условиям эксплуатации технических средств, выполняющих программу.
        \end{itemize}
    \subsection{Требования к программной документации.}
        \begin{itemize}
            \item Программная документация должна быть выполнена в соответствии с требованиями ЕСПД.
            \item На программный пакет должен быть разработан следующий комплект документации:
                \begin{itemize}
                    \item техническое задание
                    \item пояснительная записка, включающая описание языка
                \end{itemize}
        \end{itemize}
    \subsection{Требования по эргономике и эстетике.}
        \begin{itemize}
            \item У языка должен быть удобный синтаксис (минимум разделителей и скобок). Локальные определения должны быть независимы от определений в модулях.
            \item Требования к эстетике не предъявляются.
        \end{itemize}
    \subsection{Требования по маркировке, упаковке и хранению.}
        \begin{itemize}
            \item Требования по маркировке, упаковке и хранению должны соответствовать требованиям по маркировке, упаковке и хранению носителей информации, используемых для хранения программ и данных.
        \end{itemize}
    \subsection{Требования по транспортировке и хранению.}
        \begin{itemize}
            \item Требования по транспортировке и хранению должны соответствовать требованиям к транспортировке и хранению носителей информации, используемых для хранения программ и данных.
        \end{itemize}
\section{Технико-экономические требования.}
    Программный комплекс должен быть основан на бесплатном ПО с открытым исходным кодом.
\section{Стадии и этапы разработки.}
    Стадии и этапы разработки приведены в таблице: \\
    \begin{tabular}{| p{2.5cm} | p{3.3cm} | p{2.5cm} | p{2cm} | p{2cm} | p{3cm} |} \hline
        \multirow{2}{*}{Стадии работ}           & Этапы работ и их  & \multirow{2}{*}{Исполнитель}  & \multicolumn{2}{c|}{Сроки исполнения} & Результаты \\
        \hhline{~~~--~}                         & содержание        &                               & Начало   & Конец                      & работ\\ \hline
        1. Подготови-                           & 1. Разработка ТЗ  &                               & 20.09.15 & 29.09.15                   & ТЗ утверждено \\
        \hhline{~-~---}\multirow{2}{*}{тельная} & 2. Сбор информации о существующих аналогах    &   & 29.09.15 & 31.12.15                   & Обзор литературы и аналогичного ПО \\
        \hhline{~-~---}                         & 3. Изучение технической литературы по функциональным ЯП &   & 29.09.15 & 31.12.15                   & Выбор целевого ЯП и его реализации \\
        \hhline{--~---}\multirow{3}{*}{2. Основная}& 1. Разработка языка                  & Волкова & 29.09.15 & 15.03.16                   & Грамматика языка \\
        \hhline{~-~---}                         & 2. Разработка и отладка транслятора           &   & 01.03.16 & 05.04.16                   & Программная часть проекта \\
        \hhline{~-~---}                         & 3. Разработка программной документации        &   & 01.03.16 & 05.04.16                   & Описание языка \\
        \hhline{--~---}2. Заключите-            & 1. Оформление пояснительной записки           &   & 06.04.16 & 10.05.16                   & Пояснительная записка \\
        \hhline{~-~---}льная                    & 2. Создание демо-материалов к докладу         &   & 18.03.16 & 10.05.16                   & Демонстрация, примеры \\ \hline
    \end{tabular}

\section{Порядок контроля и приёмки}
    Приёмка работ осуществляется в два этапа:
        \begin{itemize}
            \item предварительная защита перед комиссией представителей кафедры ИУ-9 до 1 июня 2016 г.
            \item основная защита на Государственной Аттестационной Комиссии в период с 10 по 25 июня 2016 г.
        \end{itemize}

\section{В ходе работы отдельные пункты технического\\задания могут уточняться, дополняться,\\исключаться}

\bigskip

\begin{tabular}{ll}
    Консультант по                                                  &   Студент МГТУ \\
    конструкторской части                                           &   им. Н.Э. Баумана \\
    \underline{\hspace{3cm}}(А.В. Дубанов)                          & \underline{\hspace{3cm}}(Ю.А. Волкова) \\
    <<\underline{\hspace{0.5cm}}>>\underline{\hspace{3cm}}2015г.    & <<\underline{\hspace{0.5cm}}>>\underline{\hspace{3cm}}2015г. \\
\end{tabular}
\end{document}